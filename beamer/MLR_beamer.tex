\documentclass{beamer}

% Theme choice
\usetheme{CambridgeUS}
\usecolortheme{rose} % Colores profesionales

% Paquetes adicionales
\usepackage{graphicx} % Para insertar gráficos
\usepackage{listings} % Para insertar código
\usepackage{xcolor}   % Para personalizar colores
\usepackage{caption}  % Para personalizar captions
\usepackage{amsmath, amsthm, amssymb}
\usepackage{tcolorbox}
\tcbuselibrary{listingsutf8}

\newtheorem{ejemplo}{Ejemplo}


% Configuración de listings para código
\lstset{
    basicstyle=\ttfamily\small,
    keywordstyle=\color{blue}\bfseries,
    commentstyle=\color{gray},
    stringstyle=\color{red},
    numbers=left,
    numberstyle=\tiny,
    stepnumber=1,
    numbersep=5pt,
    frame=single,
    breaklines=true,
    showstringspaces=false,
    tabsize=4
}

% Título y detalles
\title[Relación masa-luminosidad]{Relación masa-luminosidad}
\author[Fuentes]{Á.Fuentes}
\institute[UCO]{Universidad de Córdoba}


\begin{document}

% Portada
\begin{frame}
    \titlepage
\end{frame}

% Tabla de contenidos
\begin{frame}{Índice}
    \tableofcontents
\end{frame}

% Sección 1
\section{Integación en un solo paso}
\begin{frame}{Ecuaciones diferenciales para el interior estelar}
Del capitulo 5 de los apuntes, hemos obtenido las siguientes ecuaciones diferenciales para describir el interior estelar:
\begin{enumerate}
    \item Ecuación de Equilibrio Hidrostatico
    \begin{equation}
        \frac{dP}{dr}=-\rho(r)\frac{GM_r}{r^2}
        \label{eq_hidro}
    \end{equation}
    \item Ecuación de continuidad para la masa
\begin{equation}
            \frac{dM_r}{dr}=\rho(r)4\pi r^2
            \label{masa}
        \end{equation}
    \item Para la luminosidad (ecuación para el balance de energía)
    \begin{equation}
        \frac{dL_r}{dr}=\varepsilon \rho(r)4\pi r^2
        \label{lum}
    \end{equation}
    \end{enumerate}
\end{frame}
\begin{frame}{Ecuaciones diferenciales para el interior estelar}
    \begin{enumerate}
    \setcounter{enumi}{3}
        \item Ecuación de transporte radiativo
            \begin{equation}
                \frac{dT^4}{dr}=-\rho(r)\frac{\frac{\kappa}{4\pi \sigma}L_r(r)}{r^2}
                \label{tem}
            \end{equation}
            

        \item Ecuación de transporte convectivo 
        \begin{equation}
            \frac{dT}{dr}=(1-\frac{1}{\gamma})\frac{T}{P}\frac{dP}{dr}
        \end{equation}
        \item Ecuación de estado 
        \begin{tcolorbox}
    \begin{equation}
            P(r)=\frac{k_B}{m}\rho(r)T(r)
    \end{equation}
        \end{tcolorbox}
 
    \end{enumerate}
\end{frame}

% Sección 2
\section{Relación Masa-Luminosidad}
\begin{frame}{Integración en un solo paso}
Este método consiste en hacer los siguientes cambios para resolver las ecuaciones vistas anteriormente, para las variables que aparecen en ellas:
\begin{tcolorbox}
    \begin{equation}
        X=\frac{x(r=R)-x(r=0)}{2}
    \end{equation}
\end{tcolorbox}
y los diferenciales se convierten en incrementos de forma que:
\begin{equation}
    dX\approx \Delta X=X(r=R)-X(r=0)
\end{equation}
\begin{ejemplo}[Ecuación de equilibrio hidrostático \eqref{eq_hidro}]
\begin{equation}
    P(R)-P(0)=-G\frac{\frac{M(R)+M(0)}{2}\cdot \frac{\rho(R)+\rho(0)}{2}}{(\frac{R+0}{2})^2} \rightarrow P_0=G\frac{M\rho_0}{R}
    \label{P_0}
\end{equation}
\end{ejemplo}
\end{frame}
\begin{frame}{Soluciones}
Aplicando este método llegamos a las siguientes expresiones (consideramos el ejemplo anterior para \eqref{eq_hidro}): 
\begin{itemize}
    \item Para \eqref{masa}: 
    \begin{tcolorbox}
    \begin{equation}
        M=\frac{\pi}{2}\rho_0R^3
   \end{equation}
    \end{tcolorbox}

   \item Para \eqref{lum}: 
   \begin{tcolorbox}
       \begin{equation}
    L_r=\frac{R^3\pi \varepsilon}{4}\rho_0
   \end{equation}
   \end{tcolorbox}
\end{itemize}
\end{frame}

\begin{frame}{Soluciones y expresión de la ecuación de estado}
\begin{itemize}
    \item De la ecuación \eqref{tem}
    \begin{tcolorbox}
    \begin{equation}
    L_r=\frac{R^3\pi \varepsilon}{4}\rho_0
    \label{L_r}
   \end{equation}  
    \end{tcolorbox}
\end{itemize}
    Usando estas expresiones podemos escribrir la ecuación de estado como: 
    \begin{equation}
        P_0=\frac{2 \rho_0}{m_p}k_BT_0
    \end{equation}
De la ecuación de la masa podemos despejar $\rho_0$, lo que nos permite sustituir en la expresión de $P_0$
\begin{equation}
    \rho_0=\frac{2M}{\pi R^3} \xrightarrow{\eqref{P_0}}P_0=\frac{2GM}{\pi R^4}
\end{equation}
\end{frame}

\begin{frame}
De la ecuación de estado y considerando la solución de \eqref{tem}. Tenemos
\begin{equation}
    T_0=\frac{GMm_p}{2Rk_b} \xrightarrow{\eqref{L_r}} L=\frac{\pi^2 \sigma}{k}\Bigl(\frac{Gm_p}{2k_B} \Bigl)^4M^3
\end{equation}
Lo que nos permite llegar por fin a la relación que buscamos
\begin{tcolorbox}
    \begin{equation}
        L \propto M^3
    \end{equation}
\end{tcolorbox}
Podemos usar esta relación con los valores del Sol, ya que son conocidos. de forma que: 
\begin{equation}
    \frac{L_*}{L_{\odot}}= \Bigl(\frac{M_*}{M_{\odot}}\Bigl)^3
\end{equation}

El resto de relaciones que estudiaremos a continuación son aproximadas: 
\end{frame}

\begin{frame}{Tiempo de vida nuclear}
\begin{itemize}
    \item Podemos considerar constante $T_0$, de forma que:
    \begin{equation}
        \frac{M}{R}=\frac{2T_0 k_B}{Gm_p} \approx \text{cte}
    \end{equation}
de forma que la relación radio-masa queda: 
\begin{tcolorbox}
    \begin{equation}
        R \propto M
    \end{equation}
\end{tcolorbox}
\item La relación entre la densidad (central) y la masa vendrá dada por:
\begin{equation}
    \rho_0=\frac{2M}{\pi R^3} \rightarrow \rho_0= M^{-2}
\end{equation}

\end{itemize}  
\end{frame}
\begin{frame}
\begin{itemize}
    \item La relación emisividad-masa vendrá dada por
    \begin{equation}
        \varepsilon_0=\frac{4}{\pi} \frac{L}{\rho_0 R^3} \propto \frac{1}{M^{-2}} \frac{M^3}{M^3} \rightarrow \varepsilon_0 \propto M^2
    \end{equation}
    \item Y para la relación entre la presión central con masa: 
    \begin{equation}
        P_0=\frac{2}{\pi} \frac{GM^2}{R^4} \rightarrow P_0 \propto \frac{M^2}{M^4}=M^{-2}
    \end{equation}
\end{itemize}
Llegados a este punto podemos encontrar una relación entre el tiempo de vida nuclear de una estrella y su masa: 
\begin{equation}
    t_{nuc} \propto \frac{\text{Cantidad de combustible nuclear}}{\text{Potencia con la que se gasta}} \propto \frac{M}{L} \propto \frac{M}{M^3}=\frac{1}{M^2}
\end{equation}
entonces tenemos que
\begin{tcolorbox}
    \begin{equation}
        t_{nuc} \propto M^{-2}
    \end{equation}
\end{tcolorbox}
\end{frame}
\section{Enana blanca}
\subsection{Enana blanca no relativista}
\begin{frame}{Ecuación de estado}
Las enanas blancas son un remanente estelar que se genera cuando una estrella de masa menor a $10M_{\odot}$ han agotado su combustible nuclear. Para el modelo relativista tenemos la siguiente ecuación de estado
\begin{tcolorbox}
    \begin{equation}
        P=K_1 \rho^{5/3}
    \end{equation}
\end{tcolorbox}
Aplicamos las relaciones obtenidas anteriormente como solución de \eqref{eq_hidro} y \eqref{masa} para sacar las siguientes conclusiones. De la ecuación de estado al iguala con \eqref{P_0}: 
\begin{equation}
    \frac{\rho_0GM}{R}=K_1\rho_0^{5/3}
\end{equation}
despejando $\rho_0$ para despues despejar en la ecuación de continuidad de la masa
\begin{equation}
    \rho_0=\Bigl(\frac{GM}{K_1}\Bigl)^{3/2} R^{-3/2}
\end{equation}
\end{frame}
\begin{frame}{Relación radio masa}
    \begin{equation}
        M=\frac{\pi}{2}\Bigl(\frac{GM}{K_1} \Bigl)^{3/2}R^{-3/2}R^3 \rightarrow M=\Bigl(\frac{2}{\pi} \Bigl)^2 \Bigl(\frac{K_1}{G}\Bigl)^3R^{-3} 
    \end{equation}
de donde se obtiene la relación
\begin{tcolorbox}
    \begin{equation}
        M \propto R^{-3} \rightarrow R \propto  M^{-1/3}
    \end{equation}
\end{tcolorbox}
\textbf{El radio de la estrella disminuye conforme aumenta su masa}
\end{frame}
\subsection{Enana blanca relativista y límite de Chandraseckar}
\begin{frame}{Ecuación de estado}
El razonamiento es similar al anterior, pero esta vez nuestra ecuación de estado será 
\begin{tcolorbox}
    \begin{equation}
        P_0=K_2\rho_0^{4/3}
    \end{equation}
\end{tcolorbox}
de forma que: 
\begin{equation}
    \frac{\rho_0GM}{R}=K_2\rho_0^{4/3} \rightarrow \rho_0=\Bigl( \frac{GM}{K_2R}\Bigl)^3
\end{equation}
y de la ecuación de continuidad de la masa
\begin{equation}
    M=\frac{\pi}{2} \Bigl( \frac{GM}{K_2R}\Bigl)^{3/2}R^3 \rightarrow M^2 \equiv \text{cte}
\end{equation}
Hemos llegado al valor máximo para la masa de una enana blanca: el límite de Chandrasekhar. Con valor:
\begin{equation}
    M_{CH}=\Bigl(\frac{2}{\pi} \Bigl)^{1/2} \Bigl(\frac{K_2}{G}\Bigl)^{3/2} = 1,44 M_{\odot}
\end{equation}
\end{frame}
\end{document}
