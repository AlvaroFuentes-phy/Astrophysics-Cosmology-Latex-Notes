\documentclass[openany]{book}

% =======================
% Paquetes básicos
% =======================
\usepackage[utf8]{inputenc}
\usepackage[T1]{fontenc}
\usepackage[spanish]{babel}
\usepackage{amsmath, amssymb, amsthm}
\usepackage{mathpazo}
\usepackage{graphicx}
\usepackage{xcolor}
\usepackage{tcolorbox}
\usepackage{fancyhdr}
\usepackage{geometry}
\usepackage{hyperref}
\usepackage{enumitem}
\usepackage{titlesec}
\usepackage{microtype}
\usepackage{booktabs}
\usepackage[table,xcdraw]{xcolor}
\usepackage{siunitx}
\usepackage{wasysym}
\usepackage{tikz}
\usetikzlibrary{angles, quotes, intersections, calc, shapes.geometric}

% =======================
% Configuración de página
% =======================
\geometry{margin=2.5cm}
\setlength{\parskip}{0.5em}
\setlength{\parindent}{0pt}

% =======================
% Colores personalizados
% =======================
\definecolor{ucoblue}{HTML}{003366}
\definecolor{ucogold}{HTML}{C6A664}
\definecolor{lightgray}{HTML}{F5F5F5}

% =======================
% Encabezado y pie
% =======================
\pagestyle{fancy}
\fancyhf{}
\rhead{\textbf{Apuntes para la asignatura de Astrofísica y Cosmología}}
\lhead{\includegraphics[height=1cm]{uco-logo.png}}
\setlength{\headsep}{1cm}
\cfoot{\thepage}

% =======================
% Estilo de títulos
% =======================
\titleformat{\section}
  {\normalfont\Large\bfseries\color{ucoblue}}
  {\thesection}{1em}{}

\titleformat{\subsection}
  {\normalfont\large\bfseries\color{ucogold}}
  {\thesubsection}{1em}{}

% =======================
% Entornos Teorema, Ley, etc.
% =======================
\newtheoremstyle{ucostyle}
  {10pt} % Space above
  {10pt} % Space below
  {\itshape} % Body font
  {} % Indent
  {\bfseries\color{ucoblue}} % Head font
  {.} % Punctuation after theorem name
  {0.5em} % Space after theorem name
  {} % Theorem head spec

  \titleformat{\chapter}
  {\normalfont\Huge\bfseries\color{ucoblue}}
  {\thechapter}{1em}{}

\theoremstyle{ucostyle}
\newtheorem{teorema}{Teorema}[chapter]
\newtheorem{ley}{Ley}[chapter]
\newtheorem{corolario}{Corolario}[chapter]
\newtheorem{definicion}{Definición}[chapter]
\newtheorem{ejemplo}{Ejemplo}[chapter]

% =======================
% Cajas personalizadas
% =======================
\tcbset{
  frame code={},
  center title,
  left=2mm,
  right=2mm,
  top=1mm,
  bottom=1mm,
  colback=lightgray,
  colframe=ucoblue!60!black,
  fonttitle=\bfseries,
  rounded corners,
  enhanced,
  boxrule=0.6pt
}

% Caja de fórmula
\newtcolorbox{formula}[1][]{
  colback=ucoblue!5!white,
  colframe=ucoblue!70!black,
  title=#1
}

% Caja de nota
\newtcolorbox{nota}[1][]{
  colback=ucogold!10!white,
  colframe=ucogold!70!black,
  title=#1
}

% =======================
% Documento
% =======================
\begin{document}

\begin{titlepage}
  \centering
  \vspace*{3cm}
  {\Huge \bfseries Apuntes para la asignatura de Astrofísica y Cosmología \\[0.5em]}
  {\Large Universidad de Córdoba}\\[1em]
  \vspace{15pt}
  \includegraphics[width=7cm]{Logotipo_I_Facultad_de_Ciencias_Fondo_blanco_negativo.png}\\
 
  \vspace{2cm}
  {\large Autor: Álvaro Fuentes Sánchez}\\
  {\large Profesor: Antonio José Cuesta Vázquez }
  \vspace{3cm}
  \hspace{2cm}
\begin{flushleft}
\Large\textit{«No explicar la ciencia me parece perverso. Cuando estás enamorado, quieres contárselo al mundo»}\\[0.5em]
\large--- Carl Sagan
\end{flushleft}
\end{titlepage}
\newpage

\clearpage
\thispagestyle{empty} % Quita el número de página

% --- SECCIÓN DE DERECHOS DE AUTOR ---
\begin{center}
    \Large \textbf{Derechos de autor}
\end{center}

\vspace{0.5cm}

\noindent Al ser unos apuntes de clase, este documento no pretende ser un trabajo original, sino que está basado en gran parte en las clases impartidas en la \textbf{Universidad de Córdoba} y diversos libros de texto de referencia. Para hacer la lectura más amena, se han omitido las referencias explícitas en cada sección.

\vspace{0.3cm}

\noindent Estos apuntes están escritos bajo la licencia Creative Commons, concretamente con la licencia:

\begin{center}
    \textbf{Reconocimiento-NoComercial CC-BY-NC}
\end{center}

\noindent Esto implica que:
\begin{itemize}[label=\textbullet]
    \item El beneficiario de la licencia tiene el derecho de copiar, distribuir, exhibir y representar la obra y hacer obras derivadas siempre y cuando reconozca y cite la obra de la forma especificada por el autor.
    \item El beneficiario de la licencia tiene el derecho de copiar, distribuir, exhibir y representar la obra y hacer obras derivadas para fines no comerciales.
\end{itemize}

\vspace{0.5cm}

% --- SECCIÓN DE COPYRIGHT (INGLÉS) ---
\begin{center}
    \Large \textbf{Copyright}
\end{center}

\vspace{0.5cm}

\noindent Being a set of lecture notes, this work pretends by no means to be original, but clearly relies heavily on textbooks and courses from the \textbf{University of Córdoba}. In order to make the text more readable, we have omitted the explicit references in the text.

\vspace{0.3cm}

\noindent This work has been written under the Creative Commons license, more specifically under the licence:

\begin{center}
    \textbf{Attribution-NonCommercial CC-BY-NC}
\end{center}

\noindent This means that:
\begin{itemize}[label=\textbullet]
    \item Licensees may copy, distribute, display and perform the work and make derivative works based on it only if they give the author or licensor the credits in the manner specified by these.
    \item Licensees may copy, distribute, display and perform the work and make derivative works based on it only for noncommercial purposes.
\end{itemize}

\vspace{1.5cm}

\begin{flushright}
    Fuente Palmera, \today
\end{flushright}
\tableofcontents

\clearpage
\thispagestyle{empty} % Quita el número de página

\begin{flushright}
    \textit{En memoria de mi abuela Antonia, \\
    por que fuiste la primera persona \\ en preguntarme acerca de las estrellas, \\ y ahora se que tengo una, pequeñita pero firme.
    \\
    Gracias por tanto abuela}
\end{flushright}

\clearpage



\begin{figure}
    \centering
    \includegraphics[width=1\linewidth]{portada.png}
    \caption{La galaxia espiral NGC 1055 es un miembro dominante de un pequeño grupo de galaxias ubicado a tan solo 60 millones de años luz de distancia en dirección a la imponente y acuática constelación de Cetus (la Ballena).}
\end{figure}


\chapter{Astronomia de posición. Coordenadas astronómicas}
\begin{flushleft}
\Large\textit{«Una de las cosas que me fascinan de la astronomía es la perspectiva sobre la posición en el universo. Somos tan pequeños. Hay una imagen de la tierra desde el exterior, donde observas que es un pixel. Todo lo que nos agobia, los problemas que hay, son tan pequeños»}\\[0.5em]
\large--- Gabriela Calistro Rivera
\end{flushleft}

Desde nuestros orígenes, los seres humanos hemos levantado la vista hacia las estrellas, maravillados por el cielo y las preguntas que este encierra. Los movimientos de los astros, el paso de los cometas o la salida y puesta del Sol han acompañado a la humanidad desde sus inicios. 

En este tema buscamos dar respuesta a la siguiente pregunta: \textbf{¿como podemos ubicar las estrellas en el cielo nocturno?} Las coordenadas astronómicas sirven para definir la posición de un cuerpo celeste en el cielo cuando lo observamos desde un punto en concreto. 

Como paso previo al desarrollo de estas coordenadas, vamos dar una serie de definiciones básicas. 

\begin{definicion}[Esfera celeste]
    Llamamos esfera celeste a la esfera concéntrica al globo terrestre en la que se mueven los astros. Supondremos que tiene radio $r=1$, es decir todas las estrellas se encuentran a la misma distancia. Ver Fig. \ref{fig:esfera}
\end{definicion}
\begin{definicion}[Ecuador celeste]
Se trata de la proyección del ecuador terrestre sobre la esfera celeste, y es perpendicular al eje de rotación   
\end{definicion}
\begin{definicion}[Polos celestes]
Proyección de los polos norte y sur sobre la esfera celeste
\end{definicion}
\begin{definicion}[Horizonte celeste]
Se trata de la proyección del horizonte terrestre sobre la esfera celeste, es perpendicular a la línea cenit-nadir.
\end{definicion}
\begin{definicion}[Línea Cenit-Nadir]
Es la linea imaginaria que une al punto que se encuentra justo por encima del observador (Cenit) con su punto opuesto (Nadir)
\end{definicion}
\begin{definicion}[Meridiano del lugar]
Meridiano que pasa por el punto en el que se encuentra el observador
\end{definicion}
\begin{definicion}[Latitud]
Medida angular de la distancia entre un punto y el ecuador
\end{definicion}
\begin{definicion}[Longitud]
Medida angular de la distancia entre un punto y el meridiano de Greenwich.
\end{definicion}

Estas definiciones nos ayudaran posteriormente para establecer sistemas de coordenadas que nos sirvan para localizar los objetos celestes que estudiamos

\begin{figure}
    \centering
    \includegraphics[width=1\linewidth]{esfera_celeste.png}
    \caption{Proyección estereográfica de la esfera celeste con los elementos más importantes.}
    \label{fig:esfera}
\end{figure}
\newpage

\section{Coordenadas horizontales}
El centro de este sistema de coordenadas se sitúa donde está el observador, por lo tanto estas coordenadas dependen de la ubicación del observador. En este sistema encontramos dos variables; el acimut y la altura 

\begin{definicion} [Altura]
Es el ángulo medido desde el horizonte astronómico hasta el centro del astro. Sus valor es positivo cuando el astro es visible, siendo el máximo $90^º$ (cenit) y negativo cuando se encuentra por debajo del horizonte. 
\end{definicion}

\begin{definicion}[Azimut]
Es el ángulo medido desde el sur girando en sentido horario hasta la vertical del astro, es decir, el ángulo que forma el meridiano con el cículo vertical que pasa por un punto en la esfera celeste. 
\end{definicion}

\begin{figure}
    \centering
    \includegraphics[width=1\linewidth]{image.png}
    \caption{Representación de las coordenadas horizontales}
    \label{fig:placeholder}
\end{figure}

En latitudes intermedias, los astros salen (orto) y se ponen (ocaso). En el hemisferio norte, los objetos más cercanos al Polo Norte son siempre visibles (Ver Fig. \ref{fig:circum} y \ref{fig:circump2}) mientras que otros nunca lo son (invisibles)
\\


\begin{definicion}[Culminación]
La culminación de un astro ocurre cuando este pasas por el meridiano del lugar (Es el punto más alto en su movimiento)
\end{definicion}

En la figura \ref{meridiano} se puede observar una representación de las coordenadas horizontales, marcando la culminación de una estrella.

\begin{figure}
    \centering
    \includegraphics[width=1\linewidth]{circumpolares.png}
    \caption{Fotografía de larga exposición donde apreciamos el movimiento de los astros a lo largo de un día}
    \label{fig:circum}
\end{figure}

\begin{figure}
    \centering
    \includegraphics[width=0.5\linewidth]{circump.png}
    \caption{Representación en coordenadas horizontales de las estrellas circumpolares e invisibles}
    \label{fig:circump2}
\end{figure}

\begin{figure}
    \centering
    \includegraphics[width=1\linewidth]{coordenadashor.png}
    \caption{Coordenadas horizontales para un astro $A$}
    \label{meridiano}
\end{figure}

\section{Coordenadas horarias}
Estas coordenadas sirven como paso intermedio entre horizontales y ecuatoriales. El ángulo horario es el único que avanza con el tiempo indicándonos cuanto falta para el paso el objeto por el meridiano del lugar. Definimos estos dos ángulos,

\begin{tcolorbox}
   \begin{definicion}[Declinación $\delta$]
Es el ángulo que forma la visual del astro con el plano ecuatorial. Es una coordenada universal, válida desde cualquier punto de la Tierra. Comienza en $0^º$ sobre el ecuador y alcanza $90^º$ en el polo norte celeste y $-90^º$ en el $-90^º$ en el polo sur.
\end{definicion}

\begin{definicion}[Ángulo horario $h$]
Es el ángulo que forma el meridiano del lugar con el círculo horario que pasa por el astro. Se cuenta en el sentido del movimiento diurno. Se mide en grados sexagesimales. 
\end{definicion} 
\end{tcolorbox}

La declinación es constante para cada estrella mientras que el ángulo horario depende del observador.
Podemos ver estas magnitudes representadas en la figura \ref{fig:horarias}. Veamos la diferencia con un ejemplo,
\begin{ejemplo}
    Para la estrella polar tendremos, 
    \begin{itemize}
    \item $\delta=90^º$
    \item $h$ sin definir ya que no gira a lo largo del día
    \end{itemize}
\end{ejemplo}

Todas las estrellas con declinación cercana a $90º$ son \textbf{circumpolares} (es decir, siempre son visibles), y cuando su declinación es cercana a $-90º$ son \textbf{invisibles}.


\begin{figure}[h]
    \centering
    \includegraphics[width=1\linewidth]{horarias.png}
    \caption{Representación de las coordenadas horarias para una estrellas}
    \label{fig:horarias}
\end{figure}

\textbf{Buscamos dibujar un mapa celeste}, un catalogo que posicione los objetos que podemos ver en el cielo. por lo tanto, debemos buscar unas coordenadas que sigan el movimiento diurno y en las que las estrellas tengan coordenadas que no varíen con el tiempo.  Utilizaremos unas coordenadas que usen como referencia el\textbf{ plano del ecuador celeste} y como dirección de referencia la del \textbf{punto Aries} ($\aries$). 

\section{Coordenadas ecuatoriales}
Estas coordenadas si nos sirven para etiquetar los objetos el cielo para todo lugar y todo instante de tiempo (excepto en objetos del sistema solar). 
Las coordenadas ecuatoriales son: la declinación y la ascensión recta $\alpha$. 
\begin{definicion}[Ascensión recta]
Es el arco del ecuador celeste medido en sentido directo a partir del punto Aries ($\aries$) hasta el meridiano que pasa por el astro. Varía de 0 a 24 horas. 
\end{definicion}
El sol se encuentra en el punto Aries en el equinoccio de primavera. 

\section{Relación entre las coordenadas horarias y ecuatoriales}
El tiempo sidéreo local es el ángulo horario del punto Aries. También por su definición es la ascensión recta (notese que para $H=0$, tenemos que $\alpha=TSL$) de la estrella que está pasando por el meridiano (ver Fig. \ref{TSL}). EL TSL, viene dado por
\begin{equation}
    TSL=\alpha +H
\end{equation}

Vamos a introducir una tabla (Fig.\ref{fig:tabla1}) a modo de resumen para recoger los distintos sistemas de coordenadas vistos en este tema

\begin{figure}
    \centering
    \includegraphics[width=1\linewidth]{angulos.png}
    \caption{Resumen de los distintos sistemas de coordenadas recogidos a lo largo del tema. Realizada por el profesor Antonio José Cuesta Vázquez}
    \label{fig:placeholder}
\end{figure}

\begin{figure}
    \centering
    \includegraphics[width=1\linewidth]{tabla_coordenadas.png}
    \caption{Resumen de los distintos ángulos vistos a lo largo del tema 1, sus dependencias y sus intervalos de definición. Realizada por el profesor Antonio José Cuesta Vázquez}
    \label{fig:tabla1}
\end{figure}

\begin{figure}[h]
    \centering
    \includegraphics[width=1\linewidth]{ecuac.png}
    \caption{Las coordenadas ecuatoriales se destacan en rojo; la eclíptica en amarillo; el desplazamiento aparente del Sol —no representado— se mueve imaginariamente por la eclíptica en sentido contrario a las agujas del reloj y en color azul está el primer punto de Aries por donde se pasa al norte del ecuador celeste que se considera como la primavera del hemisferio norte. Cuando después de seis meses el punto llega al Punto Libra sucede lo contrario; es decir, llega el otoño al hemisferio norte. Para el hemisferio sur, el proceso es a la inversa.}
    \label{fig:placeholder}
\end{figure}

\begin{figure}[h]
    \centering
    \includegraphics[width=1\linewidth]{TSL.png}
    \caption{Relación entre el Tiempo sidéreo local, ángulo horario y ascensión recta.}
    \label{TSL}
\end{figure}


\section{Movimientos de la Tierra}
La Tierra está sometida a movimientos de diversa índole. Los principales movimientos de la Tierra se definen con referencia al Sol y son: rotación, traslación, precesión, nutación, bamboleo de Chandler y la precesión del perihelio. Nosotros estudiaremos los siguientes. 
\subsection{Precesión de los equinoccios}

La precesión de los equinoccios (el cambio lento y gradual en la orientación del eje de rotación de la Tierra (ver Fig. \ref{fig:precesión})) se debe al movimiento de precesión de la Tierra causado por el momento de fuerza ejercido por el sistema Tierra-Sol en función de la inclinación del eje de rotación terrestre con respecto al plano de la órbita terrestre (alrededor de 23°43′ actualmente).

\begin{figure}
    \centering
    \includegraphics[width=1\linewidth]{precesión.png}
    \caption{Ángulo de precesión}
    \label{fig:precesión}
\end{figure}


\subsection{Nutación}
La precesión es aún más compleja si consideramos un cuarto movimiento: la nutación. Esto sucede con cualquier cuerpo simétrico o esferoide girando sobre su eje; un trompo (peonza) es un buen ejemplo, pues cuando cae comienza la precesión. Como consecuencia del movimiento de caída, la púa del trompo se apoya en el suelo con más fuerza, de modo que aumenta la fuerza de reacción vertical, que finalmente llegará a ser mayor que el peso. Cuando esto sucede, el centro de masa del trompo comienza a acelerar hacia arriba. El proceso se repite, y el movimiento se compone de una precesión acompañada de una oscilación del eje de rotación hacia abajo y hacia arriba, que recibe el nombre de nutación.

\begin{figure}[h]
    \centering
    \includegraphics[width=1\linewidth]{nutación.png}
    \caption{La precesión se acompaña de una oscilación del eje de rotación hacia abajo y hacia arriba, que recibe el nombre de nutación.}
    \label{fig:placeholder}
\end{figure}

\subsection{Bamboleo de Chandler}

Se trata de una pequeña oscilación del eje de rotación de la Tierra que añade 0,7 segundos de arco en un período de 433 días.

\begin{figure}
    \centering
    \includegraphics[width=1\linewidth]{transformación.png}
    \caption{Transformación de coordenadas}
    \label{fig:placeholder}
\end{figure}


\section{Transformación de coordenadas}
Vamos a trabajar usando rotaciones, así que vamos a definirlas desde un punto de vista algebraico. 
\begin{definicion}[Rotación]
Cualquier rotación es un movimiento definido en un determinado espacio que \textbf{conserva al menos un punto} en su posición original. Matemáticamente, una rotación es una \textbf{aplicación}. Todas las rotaciones sobre un punto fijo forman un grupo bajo unas reglas de composición, denominado \textbf{grupo de rotación} (de un espacio en particular).
\end{definicion}
Usualmente representamos estas rotaciones usando una matriz de rotación. 
\begin{definicion}[Matriz de rotación]
Una matriz de rotación $A$ es ortogonal y tiene determinante 1 ($\det(A)=1$) Las matrices de rotación son cuadradas y con valores reales. Sin embargo, se pueden definir sobre otros cuerpos. El conjunto de todas las matrices de rotación de dimensión $n\times n$ forma un grupo que se conoce como grupo de rotaciones (o grupo ortogonal especial).
\end{definicion}
Para el caso $n=2$, la matriz de rotación adopta la forma 
\begin{equation*}
    R(\theta)=\begin{pmatrix}
        \cos\theta &-\sin \theta
        \\
        \sin \theta & \cos \theta
    \end{pmatrix}
\end{equation*}
Para rotar vectores columnas, se multiplica por la matriz de la forma siguiente: 
\begin{equation*}
    \begin{pmatrix}
        x' \\
        y' 
    \end{pmatrix}= \begin{pmatrix}
        \cos\theta &-\sin \theta
        \\
        \sin \theta & \cos \theta
    \end{pmatrix}
    \begin{pmatrix}
        x \\
        y 
    \end{pmatrix}
\end{equation*}
Así las coordenadas despues de la rotación quedan: 
\begin{equation}
    \begin{cases}
        x'=x\cos\theta-y\sin\theta 
        \\
        y'=x\sin\theta+y\cos \theta
    \end{cases}
\end{equation}
Nosotros vamos a estudiar un caso en tres dimensiones, tenemos $\vec{x´}=A\vec{x}$, con $A \in SO(3)$ y ortogonal, estudiemos primero el caso en dos dimensiones. 
\\

Partimos de la matriz de rotación, 
\begin{equation*}
    A=\begin{pmatrix}
        \cos\theta & - \sin\theta \\\sin\theta & \cos\theta
    \end{pmatrix}
\end{equation*}
y su inversa, 
\begin{equation*}
    A^{-1}=\begin{pmatrix}
        \cos\theta &  \sin\theta \\-\sin\theta & \cos\theta
    \end{pmatrix}
\end{equation*}
\begin{nota}
    La inversa coincide con la rotación en sentido contrario
\end{nota}
Notese que
\begin{equation*}
    A^{-1}(\theta)=A(-\theta)
\end{equation*}
\\
Vamos a trabajar con una rotación en tres dimensiones, entonces si consideramos, 
\begin{equation*}
    A_{x_1}=\begin{pmatrix}
        1&0&0
        \\
        0&\cos x&\sin x
        \\
        0&-\sin x & \cos x
    \end{pmatrix}
\end{equation*}
tenemos una \textbf{rotación} \textbf{pasiva}, es decir los que rotan son los ejes del sistema de referencia.  Si denotamos por $x$ el ángulo que forman los nuevos ejes con los originales. Entonces podemos escribir la siguiente transformación,
\begin{equation*}
    \begin{pmatrix}
        x´
        \\
        y´
        \\
        z´
    \end{pmatrix}=
    \begin{pmatrix}
        1&0&0
        \\
        0&\cos x&\sin x
        \\
        0&-\sin x & \cos x
    \end{pmatrix}
    \begin{pmatrix}
        x
        \\
        y
        \\
        z
    \end{pmatrix}
\end{equation*}
y si ahora usamos coordenadas esféricas con radio $r=1$
\begin{equation*}
    \begin{pmatrix}
        \cos\psi´\cos\phi ´
        \\
        \cos \psi´\sin \phi´
        \\
        \sin\psi´
    \end{pmatrix}=
    \begin{pmatrix}
        1&0&0
        \\
        0&\cos x&\sin x
        \\
        0&-\sin x & \cos x
    \end{pmatrix}
        \begin{pmatrix}
        \cos\psi\cos\phi 
        \\
        \cos \psi\sin \phi
        \\
        \sin\psi
    \end{pmatrix}
\end{equation*}
\\

Entonces las ecuaciones de transformación quedan
\begin{align*}
    \cos\psi´\cos\phi´=\cos \psi \cos \phi
    \\
    \cos\psi´\sin \phi=\cos \psi\sin\phi \cos x +\sin \psi\sin x
    \\
    \sin\psi´=\cos\psi \sin \phi \sin x + \sin \psi \cos x 
\end{align*}
\\

\subsection{Relación entre coordenadas}

Las coordenadas horizontales se relacionan con las horarias por un giro de ángulo $x=90º-\lambda$ en torno del eje Este-Oeste.
\begin{equation*}
    \begin{cases}
        \phi=90º-a
        \\
        \psi=h
    \end{cases} \xrightarrow{x=90º - \lambda}
    \begin{cases}
        \phi´= 90º-H
        \\
        \psi= \delta
    \end{cases}
\end{equation*}
\\

Si hacemos esta sustitución en el resultado anterior y aplicando las propiedades de los ángulos complementarios tenemos un sistema de ecuaciones que nos relacionan unas coordenadas con otras
\\
\begin{formula}
\textbf{Expresiones para el paso de coordenadas horizontales a horarias}
\begin{align*}
\cos \delta \mathrm{sen}H &= \cos h \mathrm{sen}a \\
\cos \delta \cos H &= \cos h \cos a \mathrm{sen}\lambda + \mathrm{sen}h \cos \lambda \\
\mathrm{sen} \delta &= -\cos h \cos a \cos \lambda + \mathrm{sen}h \mathrm{sen}\lambda
\end{align*}
\end{formula}
y la transformación inversa
\begin{formula}
\textbf{Expresiones para el paso de coordenadas horizontales a horarias}
\begin{align*}
\cos h \mathrm{sen}a &= \cos \delta \mathrm{sen}H \\
\cos h \cos a &= \cos \delta \cos H \mathrm{sen}\lambda - \mathrm{sen}\delta \cos \lambda \\
\mathrm{sen}h &= \cos \delta \cos H \cos \lambda + \mathrm{sen}\delta \mathrm{sen}\lambda
\end{align*} 
\end{formula}

\\

Terminamos esta parte del tema estudiando la distancia angular entre dos estrellas. 
\subsubsection{Distancia angular entre dos estrellas}

La distancia angular $\theta$ corresponde al ángulo formado entre estos dos vectores. Dado que estamos en una esfera unitaria ($|\vec{r}_1| = |\vec{r}_2| = 1$), el producto escalar se simplifica a la siguiente relación:

\[
\cos \theta = \frac{\vec{r}_1 \cdot \vec{r}_2}{|\vec{r}_1| |\vec{r}_2|} = \vec{r}_1 \cdot \vec{r}_2
\]

Desarrollando el producto componente a componente:

\begin{equation*}
\cos \theta = (\cos \delta_1 \cos \alpha_1 \cos \delta_2 \cos \alpha_2) + (\cos \delta_1 \sen \alpha_1 \cos \delta_2 \sen \alpha_2) + (\sen \delta_1 \sen \delta_2)
\end{equation*}

Agrupando términos comunes y aplicando la identidad trigonométrica de la diferencia de ángulos $\cos(A-B) = \cos A \cos B + \sen A \sen B$ para las ascensiones rectas, obtenemos la fórmula fundamental:

\begin{equation}
\boxed{\cos \theta = \sen \delta_1 \sen \delta_2 + \cos \delta_1 \cos \delta_2 \cos(\alpha_1 - \alpha_2)}
\end{equation}

\section{Medida del tiempo}
El paralaje es la desviación angular de la posición aparente de un objeto, dependiendo del punto de vista elegido. Debido al acortamiento, los objetos cercanos muestran un paralaje mayor que los objetos lejanos, por lo que el paralaje puede utilizarse para determinar distancias. 
\\

El cambio de posición de una estrella en el cielo debido al movimiento anual de la Tierra en torno al Sol depende del paralaje del astro: ángulo con el que se vería el radio de la órbita terrestre en torno al Sol desde el astro estudiado.
Observemos en la imagen que ese ángulo es exactamente igual que el de la órbita aparente del astro observado desde la Tierra.
Una vez conocido el paralaje la distancia del astro al Sol se determinaría como un simple problema de Matemáticas: se trataría de calcular un cateto de un triángulo rectángulo, conociendo un ángulo (el paralaje) y el lado opuesto (la distancia de la Tierra al Sol). 


\begin{figure}[h]
    \centering
        \includegraphics[width=0.5\linewidth]{paralaje1.png}
        \label{parsec}
        \caption{Representación del paralaje entre el punto $P$ tomando como referencias dos meses del año (Enero y Julio) y un objeto celeste}
    
\end{figure}
Si $p$ es el ángulo de paralaje y $d$ la distancia desde el sol a la estrella, sabemos que $\tan(p)=1  \text{UA} /d$, de manera que si despejamos a distancia y asumismo que $p$ es un ángulo lo suficientemente pequeño podemos despejar $d$ como, 

\begin{formula}
    \begin{equation}
    d=\frac{1 \text{UA}}{p}
\end{equation}
\end{formula}
escribiendo el ángulo en radianes. 

\\

Como las distancias a las estrellas son muy grandes, conviene hacer un cambio de unidades. Si nos fijamos en Fig. \ref{fig:paralaje2} el paralaje del astro $P$ es de 1 segundo de arco.  A la distancia a la que se encuentra un astro con un paralaje de un segundo de arco se le denomina \textbf{parsec} y equivale aproximadamente a $3.26$ años luz


\begin{figure}
    \centering
    \includegraphics[width=1\linewidth]{paralaje2.png}
    \caption{Las estrellas cercanas a la Tierra parecen moverse con respecto al fondo de estrellas más distantes debido al movimiento de la Tierra alrededor del Sol, observando esa estrella con 6 meses de diferencia parece tener un fondo diferente de estrellas.}
    \label{fig:paralaje2}
\end{figure}

Terminamos el tema mostrando un mapa del cielo usando las  coordenadas ecuatoriales, este era el objetivo fundamental de este tema, poder etiquetar los objetos que vemos en el cielo. (Ver Fig. \ref{fig:mapa})

\begin{figure}
    \centering
    \includegraphics[width=1\linewidth]{mapa.png}
    \caption{Mapa del cielo en coordenadas ecuatoriales (incluye el plano de la eclíptica y el plano galáctico)}
    \label{fig:mapa}
\end{figure}




\newpage
\begin{figure}
    \centering
    \includegraphics[width=1\linewidth]{intro.png}
    \caption{G299 Remanente de una supernova de tipo Ia}
    \label{fig:placeholder}
\end{figure}

\chapter{Interacción Gravitatoria. El problema de los dos cuerpos}

\begin{flushleft}
\Large\textit{«Dejad a un lado las formas sustanciales las cualidades ocultas, y referid los hechos naturales a leyes matemáticas»}\\[0.5em]
\large--- Isaac Newton
\end{flushleft}

Kepler postulo sus tres leyes de forma observacional, posteriormente Newton haciendo uso de sus recientemente desarrollado calculo diferencial las demostró matemáticamente, enunciemos estas tres leyes. 

\begin{ley}[Primera ley de Kepler]
Los planetas describen orbitas elípticas, siendo el sol uno de los focos de la elipse
\end{ley}
Posteriormente veremos que pueden describir en general cualquier orbita cónica.
\begin{ley}[Segunda ley de Kepler]
La velocidad areolar es constante, es decir, se recorren áreas iguales en tiempos iguales
\end{ley}
\begin{ley}[Tercera ley de Kepler]
Para cualquier planeta, el cuadrado de su período orbital es directamente proporcional al cubo de la longitud del semieje mayor de su órbita elíptica.
\end{ley}



\section{Primera Ley de Kepler}
Si partimos de la primera ley de Kepler, podemos decir que de forma general las trayectorias provocadas por la interacción gravitatoria son cónicas. (ver Fig\ref{fig:conica})

\begin{figure}
    \centering
    \includegraphics[width=1\linewidth]{cónicas.png}
    \caption{Figuras cónicas, resultantes del corte de un cono con un plano}
    \label{fig:conica}
\end{figure}

Si tomamos el caso de la elipse como se ve en la Fig. \ref{fig:elipse}. Para cualquier punto $P$ de la elipse se cumple que: 
\begin{equation*}
    \overline{PF}+\overline{PF'}=\text{cte}
\end{equation*}
Para el punto $A$ de la elipse resulta que: 
\begin{equation*}
\overline{AF}+\overline{AF'}=\overline{OF}-\overline{OF'}=2\overline{OA}=\overline{AA'}=2a
\end{equation*}
es decir esto nos permite escribir 
\begin{equation*}
\overline{PF}+\overline{PF'}=\overline{AA'}=2a
\end{equation*}

\begin{definicion}[Excentricidad de la órbita]
  Podemos definir ahora la excentricidad de la órbita como,
  \begin{equation}
  \varepsilon=\frac{\overline{FF'}}{\overline{AA'}}=\frac{\overline{OF}}{\overline{OA}}
  \end{equation}
\end{definicion}


\begin{figure}
    \centering
    \includegraphics[width=1\linewidth]{elipse.png}
    \caption{Elipse, marcando ambos focos $F$ y $F'$, el centro $O$, los semiejes $A$ y $A'$ y disntintos puntos $P$}
    \label{fig:elipse}
\end{figure}

Definamos, el semieje mayor 
\begin{equation}
    \overline{OA}=a
\end{equation}
y menor de un elipse:
\begin{equation}
    \overline{OB}=b
\end{equation}
Por la definición de elipse; sabemos que
\begin{equation*}
\overline{PF}+\overline{PF'}=\overline{AA'}=2a
\end{equation*}
Para el punto que se encuentra en la perpendicular del segmento $\overline{AA'}$, llamemoslo $B$, 
\begin{equation*}
    \overline{BF}=\overline{BF'} \xrightarrow{}\overline{BF}=a
\end{equation*}
Además podemos definir la distancia del centro al foco como: \begin{equation*}
    \overline{OF}=d
\end{equation*}
y por la definición de excentricidad; 
\begin{equation*}
    d=\varepsilon a
\end{equation*}
En el triangulo rectángulo OBF podemos escribir:
\begin{equation*}
    b^2=a^2-d^2=a^2(1-\varepsilon^2)
\end{equation*}
Por lo tanto el área de la elipse vale:
\begin{formula}
\begin{equation}
    A=\pi ab= \pi a^2\sqrt{1-\varepsilon^2}
\end{equation}    
\end{formula}

\subsection{Ecuación polar de una cónica}

Llegados a este punto nos es natural hacernos las siguiente pregunta, ¿habrá una definición (y una ecuación) común para todas las cónicas?
\\


La definición de excentricidad es común para todas las cónicas; entonces, usando la fig \ref{fig:conica}:
\begin{equation*}
    \varepsilon=\frac{r}{d}=\frac{r}{\overline{FD}-r\cos\phi}
\end{equation*}
de esta forma definimos r como sigue: 
\begin{equation*}
    r=\frac{\varepsilon \overline{FD}}{1+\varepsilon\cos\phi}
\end{equation*}
simplificando llegamos a la \textbf{ecuación polar de una cónica} (Ver \ref{fig:conica}
\begin{formula}
    \begin{equation}
        r=\frac{\alpha}{1 + \varepsilon \cos \phi}
    \end{equation}
\end{formula}
donde se define $\alpha$ como 
\begin{equation*}
    \alpha=\varepsilon\overline{FD} = r_{\frac{\pi}{2}}
\end{equation*}


De esta forma para distintos valores de le excentricidad podemos diferenciar distintas orbitas, 
\begin{enumerate}
    \item Para $\varepsilon=0$, ocurre que $r=a$, y por lo tanto nuestra trayectoria es circular
    \item Para valores de $\varepsilon$ comprendidos entre cero y uno, podemos distinguir, un $r_{min}$:
    \begin{equation*}
        r_{min}=\frac{\alpha}{1 + \varepsilon}
    \end{equation*}
y un $r_{max}$ en nuestra orbita
    \begin{equation*}
        r_{max}=\frac{\alpha}{1 - \varepsilon}
    \end{equation*}
de esta forma podemos escribir
\begin{equation*}
    a=\frac{1}{2}(r_{min}+r_{máx})=\frac{\alpha}{1-\varepsilon^2}
\end{equation*}
\\
Estas trayectorias ya son abiertas por lo tanto se van al infinito para $r_{max}$
\item Por un lado; para $\varepsilon=1$ tenemos una trayectoria parabólica, con
\begin{equation*}
    r_{min}=\frac{\alpha}{2}
\end{equation*}
y como hemos indicado, se va hacia el infinito en su $r_{max}$
\item Para el caso $\varepsilon>1$, la forma es una hipérbola y nos quedaremos únicamente con la rama atractiva, entonces,
\begin{equation*}
    r_{min}=\frac{\alpha}{1 + \varepsilon}
\end{equation*}
\end{enumerate}

\subsection{Sistemas de dos partículas: Masa reducida}
Un sistema de dos masas como con el que trabajamos posee 6 grados de libertad, por lo tanto buscamos estrategias para reducirlos y facilitar la resolución del problema de los dos cuerpos. 
\\

Para todo sistema de referencia definimos la posición de su centro de masas como: 
\begin{equation}
\vec{R_{CM}}=\frac{m_1\vec{r_1}+m_2\vec{r_2}}{m_1+m_2}
\end{equation}
Y ahora podemos escribir las posiciones de cada partícula respecto a dicho centro de masas, 
\begin{align*}
    \vec{r}_{c1} &= \vec{r}_{o1} - \vec{R}_{CM}
= \left(1 - \frac{m_1}{m_1 + m_2}\right)\vec{r}_{o1}
- \frac{m_2}{m_1 + m_2}\vec{r}_{o2}
= \frac{m_2}{m_1 + m_2}(\vec{r}_{o1} - \vec{r}_{o2})
= -\frac{m_2}{m_1 + m_2}\,\vec{r},
\\[1em]
\vec{r}_{c2} &= \vec{r}_{o2} - \vec{R}_{CM}
= \left(1 - \frac{m_2}{m_1 + m_2}\right)\vec{r}_{o2}
- \frac{m_1}{m_1 + m_2}\vec{r}_{o1}
= \frac{m_1}{m_1 + m_2}(\vec{r}_{o2} - \vec{r}_{o1})
= \frac{m_1}{m_1 + m_2}\,\vec{r}.
\end{align*}

Ahora podemos definir la masa reducida del sistema como: 
\begin{formula}
    \begin{equation}
        \frac{1}{\mu}=\frac{1}{m_1}+\frac{1}{m_2} 
    \end{equation}
    o bien;
    \begin{equation}
        \mu=\frac{m_1m_2}{m_1+m_2}
    \end{equation}
\end{formula}
Y ahora haciendo uso de la masa reducida podemos escribir las posiciones de ambas particulas como
\begin{align}
    \vec{r_{C1}}=-\frac{\mu}{m_1}\vec{r}
    \\
     \vec{r_{C2}}=\frac{\mu}{m_2}\vec{r}
     \label{Relacion vectores relativos}
\end{align}
Y tanto la velocidad como la aceleración verifica para cualquiera de ellos que;
\begin{align}
\vec{v}_{c1} &= \dot{\vec{r}}_{c1} = -\,\frac{\mu}{m_1}\,\dot{\vec{v}},
&
\vec{v}_{c2} &= \dot{\vec{r}}_{c2} = \frac{\mu}{m_2}\,\dot{\vec{v}}, \\[1em]
\vec{a}_{c1} &= \ddot{\vec{r}}_{c1} = -\,\frac{\mu}{m_1}\,\ddot{\vec{r}},
&
\vec{a}_{c2} &= \ddot{\vec{r}}_{c2} = \frac{\mu}{m_2}\,\ddot{\vec{r}}.
\end{align}
Los vectores posición, velocidad y aceleración referidos al centro de masa verifican que: 
\begin{align*}
    m_1\vec{r_{1}}+m_2\vec{r_2}=0 \xrightarrow{}\vec{R_{c}}=0
    \\
     m_1\vec{v_{1}}+m_2\vec{v_2}=0 \xrightarrow{}\vec{P_{T}}=0
     \\
         m_1\vec{a_{1}}+m_2\vec{a_2}=0 \xrightarrow{}\vec{F_{T}}=0
\end{align*}

La expresión \ref{Relacion vectores relativos} entre los vectores de posición de cada partícula respecto al centro de masa y el vector posición relativa demuestra que el centro de masa está sobre la recta que une a las dos partículas (ver Fig \ref{CDM}, lo que permite establecer las relaciones entre los módulos de los vectores: 
\begin{align}
    m_1r_{C1}=m_2r_{C2}
    \\
    r_{C1}+r_{C2}=r
\end{align}

\begin{figure}
    \centering
    \includegraphics[width=0.5\linewidth]{cdm.png}
    \caption{Representación del centro de masas y de fuerzas para dos masas}
    \label{fig:placeholder}
\end{figure}

\subsection{Ecuación del movimiento}
Como en este problema sólo actúan fuerzas internas $V_{CM} \equiv \text{cte}$ y el sistema de referencia con el origen en el centro de masas es un sistema de referencia inercial.
\\

Definamos: $\vec{r}=\vec{r_2}-\vec{r_1}$. 
\\
Las fuerzas aplicadas a cada particula vienen dadas por ley gravitacional de Newton: 
\begin{equation}
    \vec{f_{12}}=G\frac{m_1m_2}{r^2}\vec{e_r}=\frac{K}{r^2}\vec{e_r}
\end{equation}
y se cumple que
\begin{equation*}
    \vec{f_{12}}=-\vec{f_{21}}
\end{equation*}

Aplicando la segunda ley de Newton podemos escribir

\[
\begin{cases}
m_1 \ddot{\vec r}_1 = -\mu \ddot{\vec r} \\[4pt]
m_2 \ddot{\vec r}_2 = \mu \ddot{\vec r}
\end{cases}
\;\longrightarrow\;
\begin{cases}
\vec f_{12} = \dfrac{K}{r^{2}}\, \vec e_r = m_1 \ddot{\vec r}_1 = -\mu \ddot{\vec r} \\[8pt]
\vec f_{21} = -\dfrac{K}{r^{2}}\, \vec e_r = m_2 \ddot{\vec r}_2 = \mu \ddot{\vec r}
\end{cases}
\]
por lo que hemos llegado a la siguiente ecuación del movimiento 
\begin{formula}
  \[
\mu \ddot{\vec r} = -\dfrac{K}{r^{2}}\, \vec e_r
\]
\end{formula}

El problema se reduce al movimiento de una partícula de masa P sometida a una fuerza
dirigida hacia el punto CF, al que llamaremos centro de fuerzas

\subsection{Momento angular y energía}
Probemos ahora dos importantes resultados, el primero de ellos, el momento angular respecto al c.d.m. vale
\[
\vec L = \mu\, \vec r \times \vec v .
\]

\begin{proof}
Tomemos el momento angular total como 
\begin{equation*}
    \vec L = \vec r_1 \times m_1 \vec v_1 + \vec r_2 \times m_2 \vec v_2  
\end{equation*}
introducimos ahora la masa reducida y llegamos a 
\begin{equation*}
     -\frac{\mu}{m_1}\, \vec r \times m_1\!\left(-\frac{\mu}{m_1}\vec v\right)
   + \frac{\mu}{m_2}\, \vec r \times m_2\!\left(\frac{\mu}{m_2}\vec v\right) = \left(\frac{\mu^2}{m_1} + \frac{\mu^2}{m_2}\right)\vec r \times \vec v = \mu^2 \left(\frac{1}{m_1} + \frac{1}{m_2}\right)\vec r \times \vec v
\end{equation*}
llegando al siguiente resultado
\begin{formula}
    \begin{equation*}
  \vec{L}=  \mu\, \vec r \times \vec v .
\end{equation*}
\end{formula}
\end{proof}

Respecto a un sistema de referencia con origen \(O\) (distinto de \(C\)):
\[
\vec L_O = \vec R_{CM} \times (m_1 + m_2)\vec V_{CM} + \mu\, \vec r \times \vec v .
\]

El siguiente resultado que demostraremos nos permite calcula la energía cinética respecto al c.d.m. usando que
\[
T = \frac{1}{2}\mu v^2 .
\]

\begin{proof} 
Partimos de la suma
\begin{equation*}
    T= \frac{1}{2} m_1 v_1'^2 + \frac{1}{2} m_2 v_2'^2
\end{equation*}
    y si ahora introducimos la masa reducida como
\begin{equation*}
    \frac{1}{2} m_1\!\left(\frac{\mu}{m_1}v\right)^2 
     + \frac{1}{2} m_2\!\left(\frac{\mu}{m_2}v\right)^2  \rightarrow  \frac{1}{2}\left(\frac{\mu^2}{m_1} + \frac{\mu^2}{m_2}\right) v^2 \rightarrow \frac{1}{2}\mu^2\left(\frac{1}{m_1} + \frac{1}{m_2}\right)v^2
\end{equation*}
usando esta ultima expresión podemos llegar a
\begin{formula}
    \begin{equation}
    T=\frac{1}{2}\mu v^2
\end{equation}
\end{formula}
\end{proof}
\\

Calculemos ahora la energía mecánica para nuestro sistema y obtengamos un par de conclusiones de estas expresiones. 
\\

La energía mecánica respecto al c.d.m. vale
\begin{formula}
\[
E = T + U = \frac{1}{2}\mu v^2 + U(r).
\]
\end{formula}


Respecto a un sistema de referencia con origen
\(O\) (distinto de \(C\)):

 \[
T_O = \frac{1}{2}(m_1 + m_2)V_{CM}^2 + \frac{1}{2}\mu v^2,
\qquad
E_O = T_O + U
     = \frac{1}{2}(m_1 + m_2)V_{CM}^2 + \frac{1}{2}\mu v^2 + U(r).
\] 



\textbf{No es necesario integrar la ecuación de movimiento para obtener algunos resultados}. Pueden obtenerse directamente a partir de los teoremas de conservación.
\\

El momento total de las fuerzas respecto al origen es nulo, luego
\[
\vec M = \vec r_1 \times \vec f_{12} + \vec r_2 \times \vec f_{21} = 0
\]
por ser \(\vec f_{ij} \parallel \vec r_{ij}\).  
Por lo tanto
\begin{formula}
\[
\vec M = \frac{d\vec L}{dt} = 0  \Rightarrow\vec L = \text{cte}.
\]
\end{formula}


Podemos hacer este otro desarrollo para demostrar directamente que \(\dot{\vec L} = \dfrac{d\vec L}{dt} = 0\):

\[
\vec L = \vec r \times \mu \vec v = \vec r \times \mu \dot{\vec r}.
\]

Derivando:
\[
\dot{\vec L}
= \dot{\vec r} \times \mu \dot{\vec r}
  + \vec r \times \mu \ddot{\vec r}
= \vec r \times \mu \ddot{\vec r}.
\]

Usando la ecuación de movimiento,
\[
\ddot{\vec r} = -\frac{K}{r^{2}}\,\vec e_r,
\]
obtenemos
\[
\dot{\vec L}
= \vec r \times \left( -\frac{K}{r^{2}} \vec e_r \right)
= -\frac{K}{r^{2}}\, \vec r \times \vec e_r = 0.
\]

De 
 \(\vec L = \mu\, \vec r \times \vec v = \text{cte}\) se sigue que  
\[
\vec r \;\text{y}\; \vec v \perp \vec L.
\]
El movimiento se desarrolla en el plano normal a \(\vec L\).
\\
Podemos hacer todo este desarrollo aplicando coordenadas polares
\[
\vec L = \mu\, \vec r \times \vec v
       = \mu\, r\vec e_r \times (\dot r\, \vec e_r + r \dot\varphi\, \vec e_\varphi)
       = \mu r^{2}\dot\varphi\, \vec e_z.
\]

Por tanto,
\[
L = \mu r^{2}\dot\varphi = \ell ,
\]
donde \(\ell\) es la componente \(L_z\) del momento angular.

El módulo vale
\[
L = |\vec L| = |\ell| > 0,
\qquad
L^{2} = \ell^{2}.
\]


\begin{figure}[h]
    \centering
    \includegraphics[width=0.5\linewidth]{momentoangular.png}
    \caption{Plano generado por el momento angular }
    \label{fig:placeholder}
\end{figure}

\section{Segunda ley de Kepler}
La segunda ley de Kepler es una consecuencia de la conservación del momento angular. 

\begin{figure}[h]
    \centering
    \includegraphics[width=1\linewidth]{segundakep-.png}
    \caption{Representación de la segunda ley de Kepler, la velocidad aerolar se mantiene constante}
    \label{fig:segundakep}
\end{figure}
Partimos de la expresión
\begin{equation*}
    dA = \frac{1}{2} |\vec{r} \times d\vec{r}| = \frac{1}{2} |\vec{r} \times \vec{v}| dt
\end{equation*}
y entonces; 
\begin{formula}
    \begin{equation}
    \frac{dA}{dt} = \frac{1}{2} |\vec{r} \times \vec{v}| = \frac{|\vec{L}|}{2\mu} = cte
\end{equation}
\end{formula}
Podemos demostrar que la energía mecánica se conserva en el problema de los dos cuerpos. 
\\
\begin{proof}
\\
Para probar la conservación, \textbf{tomamos la derivada total de la energía con respecto al tiempo}, $\frac{dE}{dt}$. Si esta derivada es cero, la energía es constante. Aplicando la regla de la cadena y la regla del producto para derivadas temporales:

\begin{equation*}
\frac{dE}{dt} = \frac{d}{dt} \left( \frac{1}{2} \mu \dot{\vec{r}} \cdot \dot{\vec{r}} - \frac{K}{r} \right) = \mu \dot{\vec{r}} \cdot \ddot{\vec{r}} + \frac{K}{r^2} \dot{r}
\end{equation*}

A continuación, introducimos la física del problema utilizando la Segunda Ley de Newton, $\vec{F} = \mu \ddot{\vec{r}}$. Una fuerza central se puede expresar de manera vectorial como $\vec{F} = -\frac{K}{r^3} \vec{r}$. Sustituimos la expresión de la fuerza en la ecuación de la derivada de la energía:

\begin{equation*}
\frac{dE}{dt} = \vec{F} \cdot \dot{\vec{r}} + \frac{K}{r^2} \dot{r} = \left( -\frac{K}{r^3} \vec{r} \right) \cdot \dot{\vec{r}} + \frac{K}{r^2} \dot{r}
\end{equation*}

Finalmente, hacemos uso de una identidad vectorial clave: $\vec{r} \cdot \dot{\vec{r}} = r \dot{r}$, la cual se deduce directamente de la derivada de $r^2 = \vec{r} \cdot \vec{r}$. Al sustituir esta identidad en la ecuación anterior, los términos resultantes se cancelan perfectamente:

\begin{equation*}
\frac{dE}{dt} = -\frac{K}{r^3} (r \dot{r}) + \frac{K}{r^2} \dot{r} = -\frac{K}{r^2} \dot{r} + \frac{K}{r^2} \dot{r} = 0
\end{equation*}

hemos demostrado que la \textbf{energía mecánica se mantiene constante en el tiempo.} 
\end{proof}

Una demostracion alternativa para este resultado se consigue usando la ecuaciónes de movimiento. Primero, recordamos las siguientes definiciones de posición 
\begin{equation*}
    \vec{r} = r\vec{e}_r
\end{equation*}
y velocidad en coordenadas polares:
\begin{equation*}
    \vec{v} = \dot{r}\vec{e}_r + r\dot{\phi}\vec{e}_\phi
\end{equation*}
 y el cuadrado de la velocidad viene dado por
 \begin{equation*}
     v^2 = \dot{r}^2 + r^2\dot{\phi}^2
 \end{equation*}


La energía total $E$ se define como la suma de la energía cinética y potencial:
\begin{equation*}
E = \frac{1}{2}\mu v^2 - \frac{K}{r} = \frac{1}{2}\mu (\dot{r}^2 + r^2\dot{\phi}^2) - \frac{K}{r}
\end{equation*}

A continuación, derivamos la energía total con respecto al tiempo para verificar si es constante:
\begin{align*}
\frac{dE}{dt} &= \frac{1}{2}\mu(2\dot{r}\ddot{r} + 2r\dot{r}\dot{\phi}^2 + 2r^2\dot{\phi}\ddot{\phi}) + \frac{K}{r^2}\dot{r} \\
&= \mu(\dot{r}\ddot{r} + r\dot{r}\dot{\phi}^2 + r^2\dot{\phi}\ddot{\phi}) + \frac{K}{r^2}\dot{r} \\
&= \mu\dot{r}\ddot{r} - \mu r\dot{r}\dot{\phi}^2 + \frac{K}{r^2}\dot{r} + \mu(2r\dot{r}\dot{\phi}^2 + r^2\dot{\phi}\ddot{\phi}) 
\end{align*}

Las ecuaciones del movimiento en coordenadas polares, derivadas de \(\mu\ddot{\vec{r}} = -\frac{K}{r^2}\vec{e}_r\), son para la componente radial: 
\begin{equation}
    \mu(\ddot{r} - r\dot{\phi}^2) = -\frac{K}{r^2} 
\end{equation}
y para la componente angular, como consecuencia de la conservación del momento angular
\begin{equation}
    \mu(2\dot{r}\dot{\phi} + r\ddot{\phi}) = 0 
\end{equation}

Multiplicamos la ecuación radial por \(\dot{r}\) y la ecuación angular por \(r\dot{\phi}\):
\begin{align*}
\mu(\ddot{r} - r\dot{\phi}^2)\dot{r} &= -\frac{K}{r^2}\dot{r} \\
\mu(2\dot{r}\dot{\phi} + r\ddot{\phi})r\dot{\phi} &= 0 \rightarrow \mu(2r\dot{r}\dot{\phi}^2 + r^2\dot{\phi}\ddot{\phi}) = 0
\end{align*}

Sustituimos estos resultados en la derivada de la energía \(\frac{dE}{dt}\):

\begin{align*}
\frac{dE}{dt} &= \left[ \mu\ddot{r} - \mu r\dot{\phi}^2 + \frac{K}{r^2} \right]\dot{r} + \left[ \mu(2r\dot{r}\dot{\phi}^2 + r^2\dot{\phi}\ddot{\phi}) \right] \\
\frac{dE}{dt} &= \left[ \mu(\ddot{r} - r\dot{\phi}^2) + \frac{K}{r^2} \right]\dot{r} + [0] \\
\frac{dE}{dt} &= \left[ -\frac{K}{r^2} + \frac{K}{r^2} \right]\dot{r} = 0
\end{align*}

comprobando que se conserva la energía mecánica durante el movimiento. 

\section{Diagramas de potencial efectivo}

La conservación de la energía mecánica nos permite utilizar los diagramas de potencial efectivo para caracterizar la trayectoria. El uso de estos diagramas es muy común y util en diversas ramas de la física. En la asignatura de Astrofísica los usaremos para caracterizar muestra orbita. El potencial efectivo se define como: $U_{\text{efc}}(r)= U_{\text{cent}}(r)+U(r)$ siendo el primer término $U_{\text{cent}}(r)$ correspondiente con el potencial centrifugo. Entonces tenemos una función dependiente de $r$, que escribimos, 
\begin{formula}
   \begin{equation}
    U_{\text{efc}}(r)=\frac{l^2}{2 \mu r^2}- \frac{K}{r}
\end{equation} 
\end{formula}
\begin{proof}
    Si escribimos la energía mecánica como
    \begin{equation*}
        E=\frac{1}{2}\mu(\dot{r}^2+r^2 \dot{\phi}^2) - \frac{K}{r} = \text{cte}
    \end{equation*}
    y sabemos que el módulo para el momento angular es $\mu r^2 \dot{\phi}=l$, tenemos una expresión de la forma 
    \begin{equation}
        E=\frac{1}{2}\mu 
        \dot{r}^2 + \frac{l^2}{\mu r^2}-\frac{K}{r}
    \end{equation}
    y entonces la energía la podemos escribir en función del potencial efectivo 
    \begin{equation*}
        E=\frac{1}{2}\mu \dot{r}^2 + U_{\text{efc}}(r)
    \end{equation*}
\end{proof}

El estudio de este función y sus puntos notables nos permite estudiar nuestra orbita. 
\begin{definicion}[Fuerza efectiva $f_{\text{efc}}(r)$]
Podemos definir la fuerza efectiva en función de r según
\begin{equation}
    f_{\text{efc}}(r)=-\frac{dU_{\text{efc}}}{dr}=\frac{l}{\mu r^3}-\frac{K}{r^2}
\end{equation}
\end{definicion}

\begin{figure}
    \centering
    \includegraphics[width=1\linewidth]{puntos_notables.png}
    \caption{Gráfica que representa, en rojo el potencial efectivo de la orbita, en azul discontinuo el potencia centrifugo y en verde el potencial real, mediante sus interpretación podemos conocer la forma de nuestra orbita}
    \label{fig:puntos_notables}
\end{figure}

\begin{figure}
    \centering
    \includegraphics[width=1\linewidth]{fuerza_efec.png}
    \caption{La fuerza efectiva es positiva para $r<r_0$ y negativa para $r>r_0$ y tiene su valor mínimo en $r_i$}
    \label{f_efec}
\end{figure}

Calculemos ahora los \textbf{puntos notables} de esta función, (ver Fig. \ref{fig:puntos_notables} y \ref{f_efec}) los mínimos vienen dado por
\begin{formula}
    \begin{equation}
        r_0=\frac{l^2}{\mu K}
\end{equation}
\end{formula}
para la posición, y por 
\begin{formula}
    \begin{equation}
     E_0=-\frac{\mu K^2}{2l^2}
\end{equation}
\end{formula}
para la energía.
\\

Escribimos los puntos de corte con el eje de abscisas, 
\begin{equation}
    r_c=\frac{l^2}{2 \mu K }=\frac{r_0}{2}
\end{equation}
y un punto de inflexión, dado por 
\begin{equation}
    r_i=\frac{3l^2}{2 \mu K}=\frac{3r_0}{2}
\end{equation}
Con estos puntos calculados escribamos las posibles trayectorias en función de la energía
\begin{figure}
    \centering
    \includegraphics[width=1\linewidth]{trayectorias.png}
    \caption{Gráfica de potencial efectivo donde se diferencia cada uno de los valores energéticos que puede tomar}
    \label{fig:placeholder}
\end{figure}
\begin{itemize}
    \item Con una energía $E=E_0<0$ se obtiene una trayectoria circular de radio (usando la expresión anterior)
    \begin{equation*}
        r_0=\frac{l^2}{\mu K }
    \end{equation*}
    como $E_0=U_{\text{efc}}(r_0)$
    entonces
    \begin{equation*}
        T_{\text{radial}}=\frac{1}{2} \mu\dot{r}^2
=E_0-U_{\text{efc}}(r_0)=0    \end{equation*}
luego se verifica que $\dot{r}=0$ y tenemos que $r=\text{cte}=r_0$. No es posible ninguna energía superior al mínimo, pues obtendríamos $\dot{r}$ que no son reales

\item Con una energía $E_0<E<0$ se obtiene una trayectoría elíptica con $r_{min}<r<r_{max}$. Tenemos que
\begin{equation*}
 T_{\text{radial}}=\frac{1}{2} \mu\dot{r}^2
=E_0-U_{\text{efc}}(r_0)>0
\end{equation*}
En los puntos de retorno $\dot{r}=0$ luego $E=V_{\text{efc}}(r)$, es decir, 
\begin{equation*}
    E=\frac{l^2}{2\mu r^2}-\frac{K}{r} \rightarrow r^2+\frac{K}{E}r-\frac{l^2}{2 \mu E}=0
\end{equation*}
y resolviendo esta ecuación de segundo grado podemos obtener $r_{min}$ y $r_{max}$, teniendo entonces
\begin{equation}
    r=-\frac{K}{2E}\Bigl(1 \pm\sqrt{1-\frac{E}{E_0}} \Bigl) = \begin{cases}
        r_{min}=-\frac{K}{2E}\Bigl(1 -\sqrt{1-\frac{E}{E_0}} \Bigl)
        \\
        r_{max}=-\frac{K}{2E}\Bigl(1 +\sqrt{1-\frac{E}{E_0}} \Bigl)
    \end{cases}
    \label{energía-semieje}
\end{equation}
de \ref{energía-semieje} podemos demostrar que el semieje mayor es solo función de $E$
\begin{formula}
   \begin{equation}
    a=\frac{1}{2}(r_{min}+r_{max})=-\frac{K}{2E}=\frac{K}{2|E|}
\end{equation} 
\end{formula}

\item Con una energía $E=0$ se obtiene una trayectoría parabólica con $r_{min}=r_c=\frac{r_0}{2}=\frac{l^2}{2\mu K}$

\item Con una energía $E>0$ se obtiene una trayectoría hiperbólica con $r_{min}$ dado por 
$r=-\frac{K}{2E}\Bigl(1\pm\sqrt{1+\frac{E}{E_0}}\Bigl)$
y nos quedamos con la solución
\begin{formula}
    \begin{equation*}
        r_{min}=\frac{K}{2E}\Bigl(\sqrt{1+\frac{E}{|E_0|}}-1\Bigl)>0
    \end{equation*}
\end{formula}
correspondiente con la primera rama de la función. Descartamos la otra solución (segunda rama) ya que la gravedad es una fuerza atractiva. 
\end{itemize}

\begin{figure}[h]
    \centering
    \includegraphics[width=0.5\linewidth]{trayectoría_hiperbólica.png}
    \caption{Trayectoria hiperbólica con parámetro de impacto $b$}
    \label{fig:placeholder}
\end{figure}
\newpage

Antes de continuar, definamos una zona en la gráfica de potencial efectivo, que nos permite caracterizar nuestro movimiento, es la \textbf{zona prohibida}

\begin{tcolorbox}
\begin{definicion}[Zona (clasicamente) prohibida]
La zona clásicamente prohibida es una región del espacio en la que, según la mecánica clásica, una partícula no puede existir ni moverse porque no tiene energía suficiente para estar allí.
Ocurre cuando la energía total de la partícula  $E$ (una recta horizontal en el diagrama) es menor que la energía potencial $U(r)$ de esa región.
\end{definicion}
\end{tcolorbox}


\section{Ecuaciones de movimiento para el problema de los dos cuerpos}
Repasemos en profundidad las ecuaciones que obtenemos para el problema de los dos cuerpos, por un lado estudiaremos la ecuación del movimiento para cada una de nuestras coordenadas generalizadas. 

\subsection{Ecuaciones de movimiento para $r$ y $\phi$}

Partimos de la ecuación de movimiento:
% Ecuación principal de movimiento, típicamente la Segunda Ley de Newton para una masa reducida (\mu)
$$ \mu \ddot{\mathbf{r}} = -\frac{k}{r^2} \mathbf{e}_r $$

\noindent Escribiéndola en coordenadas polares:
% Expresión de la posición y la primera derivada de la posición (velocidad) en polares
$$ \mathbf{r} = r \mathbf{e}_r \quad \Rightarrow \quad \dot{\mathbf{r}} = \dot{r} \mathbf{e}_r + r \dot{\phi} \mathbf{e}_{\phi} $$
% Expresión de la segunda derivada de la posición (aceleración) en polares
$$ \ddot{\mathbf{r}} = (\ddot{r} - r \dot{\phi}^2) \mathbf{e}_r + (r \ddot{\phi} + 2 \dot{r} \dot{\phi}) \mathbf{e}_{\phi} $$

\noindent Sustituyendo $\ddot{\mathbf{r}}$ en la ecuación de movimiento original y agrupando términos:
$$ \mu \ddot{\mathbf{r}} = \mu \left[ (\ddot{r} - r \dot{\phi}^2) \mathbf{e}_r + (r \ddot{\phi} + 2 \dot{r} \dot{\phi}) \mathbf{e}_{\phi} \right] = -\frac{k}{r^2} \mathbf{e}_r $$

\noindent Separamos las partes radial y acimutal (transversal):
% Utilizamos el entorno 'cases' para presentar el sistema de ecuaciones
$$
\begin{cases}
    \mu (\ddot{r} - r \dot{\phi}^2) = -\frac{k}{r^2} & (\text{Ec. mov. para } r) \\
    \mu (r \ddot{\phi} + 2 \dot{r} \dot{\phi}) = 0 & (\text{Ec. mov. para } \phi)
\end{cases}
$$

\vspace{0.5em}
La ecuación de movimiento para $\phi$ equivale a la conservación del momento angular:
% Cálculo del momento angular (\vec{L}) como producto vectorial
$$ \vec{L} = \mu \vec{r} \times \vec{v} =
    \begin{vmatrix} % Matriz para el determinante del producto vectorial
    \mathbf{e}_r & \mathbf{e}_{\phi} & \mathbf{e}_z \\
    r & 0 & 0 \\
    \dot{r} & r \dot{\phi} & 0
    \end{vmatrix}
    = \mu r^2 \dot{\phi} \mathbf{e}_z = \ell \mathbf{e}_z
$$ 
Despejando la velocidad radial $\dot{\phi}$:
\begin{formula}
$$ \dot{\phi} = \frac{\ell}{\mu r^2} $$    
\end{formula}


Se sustituye en la ecuación de movimiento para $r$:
$$ \dot{\phi} = \frac{\ell}{\mu r^2} \quad \Rightarrow \quad \dot{\phi}^2 = \frac{\ell^2}{\mu^2 r^4} $$

La ecuación de movimiento radial es:
$$ \mu \ddot{r} - \mu r \dot{\phi}^2 = -\frac{k}{r^2} $$

Sustituyendo $\dot{\phi}^2$:
$$ \mu \ddot{r} - \mu r \left( \frac{\ell^2}{\mu^2 r^4} \right) = -\frac{k}{r^2} $$
si simplificamos obtenemos
$$ \mu \ddot{r} - \frac{\ell^2}{\mu r^3} = -\frac{k}{r^2} $$

Finalmente, despejando el término $\mu \ddot{r}$ 
\begin{formula}
    $$ \mu \ddot{r} = \frac{\ell^2}{\mu r^3} - \frac{K}{r^2} $$
\end{formula}


\subsection{Ecuación de la trayectoria $r=r(\phi)$}

\noindent Partimos de la conservación del momento angular ($\ell$ es constante):
\begin{formula}
$$ \dot{\phi} = \frac{d\phi}{dt} = \frac{\ell}{\mu r^2} $$
\end{formula}

\subsubsection*{Cálculo de la primera derivada temporal $\dot{r}$}

Aplicamos la regla de la cadena para cambiar la variable de derivación de $t$ a $\phi$:
$$ \dot{r} = \frac{dr}{dt} = \frac{dr}{d\phi} \frac{d\phi}{dt} = \frac{dr}{d\phi} \dot{\phi} $$

Sustituimos la expresión del momento angular $\dot{\phi} = \frac{\ell}{\mu r^2}$:
$$ \dot{r} = \frac{dr}{d\phi} \left( \frac{\ell}{\mu r^2} \right) = \frac{\ell}{\mu r^2} \frac{dr}{d\phi} $$

$$ \dot{r} = \frac{dr}{dt} = \frac{dr}{d\phi} \frac{d\phi}{dt} = \dot{\phi} \frac{dr}{d\phi} = \frac{\ell}{\mu r^2}  \frac{dr}{d\phi} $$

\vspace{0.5cm}
\hrule

\subsubsection*{Cálculo de la segunda derivada temporal $\ddot{r}$}

La segunda derivada es la derivada temporal de $\dot{r}$:
$$ \ddot{r} = \frac{d}{dt} \left( \dot{r} \right) = \frac{d}{dt} \left( \frac{dr}{dt} \right) $$

Nuevamente, aplicamos la regla de la cadena para cambiar la derivación de $t$ a $\phi$:
$$ \ddot{r} = \frac{d}{d\phi} \left( \dot{r} \right) \frac{d\phi}{dt} = \frac{d\phi}{dt} \frac{d}{d\phi} \left( \dot{r} \right) = \dot{\phi} \frac{d}{d\phi} \left( \dot{r} \right) $$

Sustituimos $\dot{\phi}$ y la expresión de $\dot{r}$ obtenida anteriormente:
$$ \ddot{r} = \left( \frac{\ell}{\mu r^2} \right) \frac{d}{d\phi} \left( \frac{\ell}{\mu r^2} \frac{dr}{d\phi} \right) $$

Dado que $\frac{\ell}{\mu}$ es una constante, se puede extraer de la derivada en $\phi$:
$$ \ddot{r} = \frac{\ell}{\mu r^2} \cdot \frac{\ell}{\mu} \frac{d}{d\phi} \left( \frac{1}{r^2} \frac{dr}{d\phi} \right) = \frac{\ell^2}{\mu^2 r^2} \frac{d}{d\phi} \left( \frac{1}{r^2} \frac{dr}{d\phi} \right) $$

Esta expresión se simplifica  mediante el cambio de variable $u = 1/r$.\\
Calculamos la derivada de $r$ con respecto a $\phi$:
$$ \frac{dr}{d\phi} = \frac{d}{d\phi} \left( \frac{1}{u} \right) = -\frac{1}{u^2} \frac{du}{d\phi} $$

Sustituyendo $u=1/r$:
$$ \frac{dr}{d\phi} = -r^2 \frac{du}{d\phi} $$

Ahora, sustituimos esta expresión en el término $\left( \frac{1}{r^2} \frac{dr}{d\phi} \right)$ dentro de la segunda derivada:
$$ \frac{1}{r^2} \frac{dr}{d\phi} = \frac{1}{r^2} \left( -r^2 \frac{du}{d\phi} \right) = - \frac{du}{d\phi} $$

Finalmente, sustituimos $- \frac{du}{d\phi}$ en la expresión para $\ddot{r}$:
\begin{equation*}
    \ddot{r} = \frac{\ell^2}{\mu^2 r^2} \frac{d}{d\phi} \left[ \frac{1}{r^2} \frac{dr}{d\phi} \right] \rightarrow \frac{\ell^2}{\mu^2 r^2} \frac{d}{d\phi} \left[ - \frac{du}{d\phi} \right] \rightarrow  - \frac{\ell^2}{\mu^2 r^2} \frac{d^2 u}{d\phi^2}
\end{equation*}

Como $u = 1/r$, tenemos:
\begin{formula}
$$ {\ddot{r} = - \frac{\ell^2}{\mu^2 r^2} \frac{d^2}{d\phi^2} \left( \frac{1}{r} \right)} $$    
\end{formula}

Partimos de la ecuación de movimiento radial y sustituimos la expresión para la aceleración $\ddot{r}$ obtenida por cambio de variable angular:
$$ \ddot{r} = -\frac{\ell^2}{\mu^2 r^2} \frac{d^2}{d\phi^2} \left( \frac{1}{r} \right) \xrightarrow{\mu\ddot{r}=F_r} \mu \left( -\frac{\ell^2}{\mu^2 r^2} \frac{d^2}{d\phi^2} \left( \frac{1}{r} \right) \right) = \frac{\ell^2}{\mu r^3} - \frac{K}{r^2}$$ 

\noindent Simplificando $\mu$ en el lado izquierdo y reordenando la ecuación a la forma mostrada en el paso intermedio:
$$
-\frac{\ell^2}{\mu r^2} \frac{d^2}{d\phi^2} \left( \frac{1}{r} \right) - \frac{\ell^2}{\mu r^3} + \frac{K}{r^2} = 0
$$

\noindent Multiplicamos la ecuación por el factor $\frac{\mu r^2}{\ell^2}$ para despejar la derivada de segundo orden:
$$
\frac{d^2}{d\phi^2} \left( \frac{1}{r} \right) + \frac{1}{r} - \frac{\mu K}{\ell^2} = 0
$$

Ahora hacemos una serie de cambios de variable para simplificar nuestra ecuación diferencial, en primer lugar: 
\begin{formula}
    \begin{equation}
        y=\frac{1}{r}
    \end{equation}
\end{formula}

\noindent Sustituyendo $y$:
$$
\frac{d^2 y}{d\phi^2} + y - \frac{\mu K}{\ell^2} = 0
$$
Se aplica un segundo cambio de variable para eliminar el término constante y simplificar la EDO:
$$ z = y - \frac{\mu K}{\ell^2} $$

\noindent Dado que $\frac{\mu K}{\ell^2}$ es una constante, sus derivadas son cero. La sustitución de $y = z + \frac{\mu K}{\ell^2}$ en la ecuación anterior produce:
$$
\frac{d^2 z}{d\phi^2} + z = 0
$$

de aquí obtenemos la siguiente solución general; 
\begin{formula}
    \begin{equation*}
        z=A\cos(\phi-\phi_0)
    \end{equation*}
\end{formula}

\noindent \textbf{1. Deshaciendo el cambio $z$:} Recordamos que se definió como $z = y - \frac{\mu K}{\ell^2}$. 
\\
Sustituyendo la solución de $z$:
$$ y - \frac{\mu K}{\ell^2} = A \cos(\phi - \phi_0) $$

\noindent \textbf{2. Deshaciendo el cambio $y$:} Recordamos que  se definió como el inverso del radio, $y = \frac{1}{r}$. \\
Despejando $\frac{1}{r}$ se obtiene:
$$ \frac{1}{r} = \frac{\mu K}{\ell^2} + A \cos(\phi - \phi_0) $$

\noindent Para obtener la expresión del radio $r$, se calcula el inverso de la ecuación anterior y se factoriza el término constante en el denominador para obtener la forma canónica de la ecuación de una cónica, entonces:
\begin{equation*}
     r = \frac{1}{\frac{\mu K}{\ell^2} + A \cos(\phi - \phi_0)}
\end{equation*}
si sacamos factor común $\frac{\mu K}{l^2}$ tenemos que
\begin{equation*}
    r= \frac{1}{\frac{\mu K}{\ell^2} \left[ 1 + \frac{A}{\mu K / \ell^2} \cos(\phi - \phi_0) \right]}
\end{equation*}
y pasando este factor común al numerador tenemos que
\begin{equation*}
    r= \frac{\ell^2/\mu K}{1 + \left( \frac{A \ell^2}{\mu K} \right) \cos(\phi - \phi_0)}
\end{equation*}

Esta expresión es la ecuación de una cónica en coordenadas polares. Para simplificarla, se definen dos constantes fundamentales de la órbita: el parámetro $\alpha$ (semi-latus rectum)
\begin{formula}
\begin{equation}
    \alpha = \frac{\ell^2}{\mu K}
    \label{semi_latus}
\end{equation}    
\end{formula}
y la excentricidad  $\varepsilon$: 
\begin{formula}
 \begin{equation}
      \varepsilon = \frac{A \ell^2}{\mu K}
      \label{excen}
 \end{equation}    
\end{formula}

\noindent Sustituyendo los parámetros en la expresión del radio, se obtiene la forma final:
$$ r = \frac{\alpha}{1 + \varepsilon \cos(\phi - \phi_0)} $$

\noindent Si se elige el origen angular ($\phi=0$) de modo que coincida con la posición del perihelio (radio mínimo), se establece la condición $\phi_0 = 0$.

\begin{formula}
    $$ r = \frac{\alpha}{1 + \varepsilon \cos \phi} $$
\end{formula}
Veamos un rápido ejemplo de como usar estos conceptos para demostrar la relación entre el semieje mayor y la energía de la orbita
\begin{ejemplo} Demuestre que la energía de una orbita es función únicamente de su semieje mayor, usando el cálculo de los puntos de retorno: 

Partimos de la siguiente definición que relaciona la energía total con la energía potencial efectiva
\begin{equation}
    E_T=T+U_{ef}
\end{equation}
en los puntos de retorno se cumple que $T=0$, entonces tenemos que:
\begin{equation*}
    E_T=U_{ef}=\frac{l^2}{2\mu r^2}-\frac{K}{r} \rightarrow E_t-\frac{l^2}{2\mu r^2}+\frac{K}{r}=0
\end{equation*}
y ahora buscamos obtener una ecuación de segunda grado que quedara de la forma
\begin{equation}
    \label{eq_2}
    2\mu r^2E_T+2\mu Kr-l^2=0
\end{equation}
y resolvemos de forma que 
\begin{equation*}
    r=\frac{-K}{2E_T}\Bigl(1\pm\sqrt{1-\frac{E}{E_0}} \Bigl)
    = \begin{cases}
        r_{min}=-\frac{K}{2E}(1-1\sqrt{1-\frac{E}{E_0}})
        \\
        r_{max}=-\frac{K}{2E}(1+\sqrt{1-\frac{E}{E_0}})
    \end{cases}
\end{equation*}
Y ahora basta con escribir el semieje mayor como 
\begin{formula}
    \begin{equation*}
    a=\frac{1}{2}(r_{min}+r_{max})=\frac{-K}{2E} \rightarrow   E=\frac{-K}{2a}
\end{equation*}
\end{formula}
\end{ejemplo}

\section{Tercera ley de Kepler}
\noindent Un planeta completa una órbita elíptica en un tiempo $T$ (periodo).
Sea $A$ el área de la elipse. Geométricamente, el área de la elipse vale:
$$ A = \pi a b = \pi a^2 \sqrt{1-\varepsilon^2} $$
\noindent Donde $a$ es el semieje mayor, $b$ es el semieje menor, y $\varepsilon$ es la excentricidad, siendo la relación entre los semiejes:
$$ b = a\sqrt{1-\varepsilon^2} $$

\noindent Por la conservación de la \textbf{velocidad areolar}  el área de la elipse también se relaciona con el periodo $T$ y el momento angular $L$:
$$ A = T \frac{dA}{dt} = \frac{T L}{2 \mu} $$

\noindent Igualando ambas expresiones para el área $A$:
$$ \pi a^2 \sqrt{1-\varepsilon^2} = \frac{T L}{2 \mu} $$

\noindent Elevando al cuadrado esta relación, obtenemos:
\begin{formula}
    \begin{equation}
   \pi^2 a^4 (1-\varepsilon^2) = \frac{T^2 L^2}{4 \mu^2} \quad  
\label{ecuacion1}
\end{equation}
\end{formula}


\noindent Para la elipse, el semieje mayor $a$ se relaciona con el parámetro $\alpha$ y la excentricidad $\varepsilon$ mediante la siguiente relación geométrica:
$$ a = \frac{\alpha}{1-\varepsilon^2} \quad $$

\noindent Siendo $\alpha$ el parámetro dinámico definido como:
$$ \alpha = \frac{L^2}{\mu K} \quad $$

\noindent De donde resulta el semieje mayor $a$ al sustituir $\alpha$:

$$ a = \frac{L^2}{\mu K (1-\varepsilon^2)} $$

\noindent Reordenando esta expresión para despejar el término $\frac{L^2}{\mu}$, se obtiene,

\begin{formula}
  \begin{equation}
   \frac{L^2}{\mu} = a K (1-\varepsilon^2) \quad 
 \label{ecuacion2} 
\end{equation}  
\end{formula}


\noindent Sustituimos \ref{ecuacion2} en \ref{ecuacion1}:

\begin{equation*}
  \pi^2 a^4 (1-\varepsilon^2) = \frac{T^2}{4 \mu^2} \left[ \mu K a (1-\varepsilon^2) \right]  
\end{equation*}



\noindent Simplificando los términos $a$, $\mu$, y $(1-\varepsilon^2)$ en ambos lados de la ecuación y reordenando, se obtiene:
$$
\frac{a^3}{T^2} = \frac{K}{4 \pi^2 \mu}
$$


\noindent Sustituyendo las definiciones de la masa reducida ($\mu$) y la constante ($K$):
$$
\mu = \frac{m_1 m_2}{m_1 + m_2}, \quad K = G m_1 m_2
$$

Se obtiene la Tercera Ley de Kepler en su forma general:
\begin{formula}
\begin{equation*}
    \frac{a^3}{T^2} = \frac{G (m_1 + m_2)}{4 \pi^2}
\end{equation*}
\end{formula}

\vspace{0.5cm}

Podemos simplificar esta ley al aplicarla al sistema solar si se asume que la masa central es mucho mayor que la del cuerpo orbitante ($m_1 = M_{\odot}$, $m_2 = m_p$ y $M_{\odot} \gg m_p$), entonces $m_1 + m_2 \approx M_{\odot}$. La expresión se simplifica a:
$$ \frac{a^3}{T^2} \approx \frac{G M_{\odot}}{4 \pi^2} $$




\begin{figure}[h]
    \centering
    \includegraphics[width=\linewidth]{relación_tercera_ley.png}
    \caption{Ilustración de la relación entre el radio orbital y el período orbital}
    \label{fig:placeholder}
\end{figure}

\newpage
\begin{figure}
    \centering
    \includegraphics[width=1\linewidth]{leo.png}
    \caption{Las Pléyades o Las siete hermanas (Messier 45 o M45) es un asterismo de un cúmulo estelar abierto que contiene estrellas calientes de tipo espectral B, de corta edad, ubicadas en la constelación de Tauro. Está entre los cúmulos estelares más cercanos a la Tierra, y es el cúmulo más visible a simple vista en el cielo nocturno. El cúmulo está dominado por estrellas calientes extremadamente azules y luminosas que se han formado en los últimos 120 millones de años. El polvo que forma una débil nebulosidad de reflexión alrededor de las estrellas más brillantes se pensó en un principio que provenía de una disgregación de la propia formación del cúmulo (de ahí el nombre alternativo para nebulosa Maia en vez de estrella Maia), pero ahora se sabe que es una nube de polvo no relacionada en el medio interestelar, a través de la cual las estrellas están pasando actualmente.}
    \label{fig:placeholder}
\end{figure}

\chapter{Instrumentación y medidas astronómicas.}

\begin{flushleft}
\Large\textit{«Sería muy triste ser un átomo en el universo sin los físicos. Y los físicos están hechos de átomos. Un físico es la forma de un átomo de saber que hay átomos.»}\\[0.5em]
\large--- George Wald
\end{flushleft}


La información que nos llega del cosmos es mayoritariamente radiativa, aunque cada vez, tras su primera detección en el año 2015, las ondas gravitaciones son más importantes en el estudio de la astrofísica, en este tema vamos a estudiar aspectos referentes a la radiación que nos llega de las estrellas. 

\\

Las radiaciones electromagnéticas pueden definirse como aquellos procesos en los que se emite energía bajo la forma de ondas o partículas materiales y pueden propagarse tanto a través de un medio material como en el vacío.
Se diferencian unas de otras en el valor de su frecuencia. Cuanto mayor es la frecuencia de una radiación, mayor es su energía.

Las radiaciones electromagnéticas se clasifican mediante el espectro electromagnético de frecuencias, como se ve en la tabla \ref{tab:espectroEM2}




\begin{table}[h!]
\centering
\renewcommand{\arraystretch}{1.45}
\setlength{\tabcolsep}{14pt}

\caption{\textbf{Espectro electromagnético}}
\label{tab:espectroEM2}

\resizebox{0.95\linewidth}{!}{%
\begin{tabular}{|>{\centering\arraybackslash}m{3.5cm}|
                >{\centering\arraybackslash}m{4cm}|
                >{\centering\arraybackslash}m{4cm}|
                >{\centering\arraybackslash}m{4cm}|}
\hline

\textbf{Región} 
& \textbf{Longitud de onda} 
& \textbf{Frecuencia (Hz)} 
& \textbf{Energía del fotón (eV)} \\
\hline

\textbf{\textcolor{blue}{Rayos gamma}} 
& $< 0.02$ nm
& $>15$ EHz
& $>62.1$ keV \\ \hline

\textbf{\textcolor{blue}{Rayos X}}
& $0.01$–$10$ nm
& $30$ EHz – $30$ PHz
& $124$ keV – $124$ eV \\ \hline

\textbf{\textcolor{blue}{Ultravioleta}}
& $10$–$400$ nm
& $30$ PHz – $750$ THz
& $124$ eV – $3$ eV \\ \hline

\textbf{\textcolor{blue}{Visible}}
& $390$–$750$ nm
& $770$ THz – $400$ THz
& $3.2$ eV – $1.7$ eV \\ \hline

\textbf{\textcolor{blue}{Infrarrojo}}
& $750$ nm – $1$ mm
& $400$ THz – $300$ GHz
& $1.7$ eV – $1.24$ meV \\ \hline

\textbf{\textcolor{blue}{Microondas}}
& $1$ mm – $1$ m
& $300$ GHz – $300$ MHz
& $1.24$ meV – $1.24$ μeV \\ \hline

\textbf{\textcolor{blue}{Radio}}
& $1$ m – $100{,}000$ km
& $300$ MHz – $3$ Hz
& $1.24$ μeV – $12.4$ feV \\ \hline

\end{tabular}
}
\end{table}

La radiaciones de alta frecuencia (superiores a $10^{17} \text{Hz}$ son ionizantes y cuando interaccionan con la materia producen la ionización de los átomos de la misma, debido a este efecto, este tipo de radiación si supone un peligro. 

\section{Radiación del cuerpo negro} 
\begin{tcolorbox}
    Se llama cuerpo negro a aquel que absorbe toda la radiación que incide sobre él, y la que emite es solo función de su temperatura y de la frecuencia de onda.
\end{tcolorbox}
 
\\
Partiendo de la expresión del flujo específico, $F_v$ : 
\begin{equation}
    F_v=\frac{2\pi h}{c^2}\frac{v^3}{e^{\frac{hv}{K_BT}}-1}
    \label{Flujo_e}
\end{equation}

Podemos definir el flujo total si consideramos todo el espectro de frecuencias, 
\begin{equation}
    F=\int_0^\infty F_v dv
    \label{flujo_total}
\end{equation}
Por lo tanto si resolvemos \ref{flujo_total}, tenemos, 

\begin{equation*}
    F=\frac{2\pi h}{c^2}\int_0^\infty \frac{v^3}{{e^{\frac{hv}{K_BT}}-1}} dv =\frac{2\pi k_B^4}{c^2h^3}T^4\int_0^\infty \frac{x^3}{e^x-1} = \sigma T^4
\end{equation*}

usando el cambio de variable $x=\frac{hv}{K_BT}$ y tomando que la integral en x es $\frac{\pi^4}{15}$

Hemos llegado a la ley de Stefan-Boltzmann: 
\begin{formula}
    \begin{equation}
    F=\sigma T^4
        \label{S_B}
    \end{equation}
\end{formula}

El término $\sigma$ es la constante de Stefan-Boltzmann, que se define como: 
\begin{equation}
\sigma=\frac{2\pi^5k_B^4}{15c^2h^3} = 5.6693 \times 10^{-8} \si{\watt\per\meter\squared\per\kelvin\tothe{4}}
    \label{cosntS_B}
\end{equation}

Hemos encontrado una expresión que relaciona el flujo que emite un cuerpo negro con su temperatura, esta expresión nos va a ser de gran utilidad por que supondremos que las estrelas son cuerpos negros. 

\begin{figure} [h]
    \centering
    \includegraphics[width=1\linewidth]{cuerpo_negrog.png}
    \caption{En la figura observamos como varia la intensidad de la emisión en funcióm de la longitud de onda, la linea azul une los distintos picos de emisión y vemos como se van desplazando a medida que la temperatura aumenta, como explica la ley de desplazamiento de Wien}
    \label{cuerpo_negro}
\end{figure}

\section{Ley de Desplazamiento de Wien}
Esta ley nos dice que la intensidad de máxima emisión para un cuerpo negro depende de la temperatura del mismo. Para un cuerpo negro a una temperatura T, podemos usar la siguiente expresión para calcular su $\lambda_{max}$ 

\begin{formula}
    \begin{equation}
        \lambda_{max}T=\frac{hc}{5k_b}
    \end{equation}
\end{formula}

\begin{proof}
Si partimos de la expresión \ref{Flujo_e}, podemos calcular su máximo como
\begin{equation*}
    \frac{dF_\lambda}{d\lambda}=0 
\end{equation*}
Calculamos dicha derivada de forma que: 
\begin{equation*}
        \frac{dF_\lambda}{d\lambda} = \frac{-10 \pi hc^2}{\lambda^6 (e^{\frac{hc}{k_bT\lambda}}-1)}+ \frac{2\pi hc^2}{\lambda^5}\bigg( \frac{\frac{hc}{kT\lambda^2}e^{\frac{hc}{k_bT \lambda}}}{(e^\frac{hc}{k_b T \lambda}-1)^2}\bigg)
\end{equation*}
Ahora igualamos a cero, 
\begin{equation*}
    \frac{\pi h c^{2}}{\lambda^{6} (e^{\frac{hc}{kT\lambda}} - 1)}
\left[
\frac{2 h c e^{\frac{hc}{kT\lambda}}}{kT\lambda (e^{\frac{hc}{kT\lambda}} - 1)} - 10
\right] = 0
\end{equation*}
si despejamos; 
\begin{equation*}
    \frac{2 h c e^{\frac{hc}{kT\lambda}}}{kT\lambda (e^{\frac{hc}{kT\lambda}} - 1)} = 10
\end{equation*}
y llegamos a: 
\begin{equation*}
    \lambda_{\text{max}} T =
\frac{h c e^{\frac{hc}{kT\lambda}}}{5k_ b\left( e^{\frac{hc}{kT\lambda}} - 1 \right)}
\end{equation*}
y ahora en el segundo miembro de la expresión despreciamos el uno frente a la exponencial y por tanto, llegamos a: 
\begin{equation*}
    \lambda_{max}=\frac{hc}{5k_b}
\end{equation*}
\end{proof}

\begin{tcolorbox}
\begin{definicion}[Temperatura del color]
    Definimos la  \textbf{temperatura} \textbf{de color }como aquella temperatura del cuerpo negro que mejor se ajusta al espectro emitido por una fuente de luz.
\end{definicion}    
\end{tcolorbox}




\section{Luminosidad y magnitud estelar}

\begin{definicion} [Luminosidad]
   Definimos la luminosidad como la energía emitida por unidad de tiempo por una estrella de radio $R$ y temperatura superficial $T$ 
\end{definicion}

\begin{formula}
    \begin{equation}
        L=4\pi R^2F \xrightarrow{F=\sigma T^4}L=4 \pi R^2 \sigma T^4
    \end{equation}
Donde para llegar a la expresión anterior hemos supuesto que al estrella se comporta como un cuerpo negro
\end{formula}

\begin{definicion}[Luminosidad aparente o brillo de una estrella]
   Definamos ahora la luminosidad aparente o brillo de una estrella, que, de forma similar a la luminosidad, se trata del flujo de energía recibido en la superficie de la Tierra a una distancia $d$ de la estrella. 
\end{definicion}


\\

Si no hay perdida de energía podemos considerar: 
    \begin{align*}
        L=4\pi R^2 F = 4 \pi d^2 f
    \end{align*}
despejemos $f$ de forma que
\begin{equation*}
     f= \frac{R^2}{d^2}F
\end{equation*}
por lo tanto se establece una relación de proporcionalidad de forma que 
\begin{formula}
\begin{equation}
      f \propto d ^{-2}
\end{equation}
\end{formula}
Hemos llegado a esta expresión suponiendo que la estrella se comporta como un cuerpo negro. Pero una estrella no se comporta exactamente como un cuerpo negro, sino que tenemos que definir una temperatura equivalente a la de este

\begin{tcolorbox}
\begin{definicion}[Temperatura efectiva $T_e$]
$T_e$ es la temperatura de un cuerpo negro con la que se obtendría la misma luminosidad $L$
\end{definicion}    
\end{tcolorbox}

Hablemos ahora de la medida del brillo de las estrellas. En el siglo II A.C, Hiparco de Nicea establece 6 magnitudes de forma únicamente observacional. Siendo la 1ª magnitud más brillante y descendiendo hasta la 6ª. No fue hasta el año 1856 que N.Pogson estableció la escala moderna de magnitudes, de forma que:
\begin{equation*}
    \text{Brillo (1ªmag)}= 100 \times \text{Brillo (6ªmag)}
\end{equation*}
Entonces tenemos que; 
\begin{equation}
    \sqrt[5]{100}=2,512
    \label{2,512}
\end{equation}

Podemos relacionar la diferencia de magnitudes con el cociente de brillos. Del resltado \ref{2,512} podemos escribri la siguiente expresión: 
\begin{equation*}
    \frac{f_1}{f_2}=\big( \sqrt[5]{100}\big)^{m_2-m_1}
\end{equation*}
Podemos simplificar dicha expresión y llegar a: 
\begin{equation}
    m_1-m_2=-2,5\log\bigg(\frac{f_1}{f_2} \bigg)
\end{equation}

Además, esta expresión nos permite definir la magnitud aparente y absoluta de una estrella. 

\\ 

Tomando como referencia una estrella de magnitud cero, la magnitud aparente de una estrella vendrá dada por
\begin{equation}
    m=-2,5 \log\bigg(\frac{f}{f_0}\bigg)
\end{equation}
siendo el término $f_0$ el brillo que  corresponde a $m_0=0$. 

\\

Conociendo el concepto de magnitud aparente podemos definir la magnitud absoluta como la magnitud aparente de la estrella si estuviera situada a una distancia de $10 pc$. Por lo tanto, si $f$ es el flujo recibido a una distancia de $d$ parsec y $f_{10}$, podemos considerar: 
\begin{equation*}
    \frac{f}{f_{10}}=\bigg(\frac{10}{d} \bigg)^2
\end{equation*}
Y podemos escribir; 
\begin{align*}
    m-M=-2,5 \log\bigg( \frac{f}{f_{10}}\bigg)=-2,5 \log \bigg( \frac{10}{d}\bigg)^2 = -5 \log \bigg( \frac{10}{d}\bigg)= 5 \log \bigg( \frac{d}{10}\bigg)
\end{align*}
De forma que, podemos llegar a la siguiente expresión que relaciona magnitudes y distancias, 
\begin{formula}
    \begin{equation}
        M=m-5 \log \bigg(\frac{d}{10} \bigg)
    \end{equation}
\end{formula}
Se llama módulo de la distancia a la diferencia 
\begin{equation*}
    m-M=5 \log\bigg( \frac{d}{10}\bigg)
\end{equation*}

Resumiendo, 
\begin{figure}[h]
    \centering
    \includegraphics[width=1\linewidth]{resumen_t3.1.png}
    \caption{Resumen y comparativa de las principales magnitudes introducidas en este capitulo de los apuntes. Realizado por el profesor Antonio Jose Cuesta Vázquez}
    \label{fig:placeholder}
\end{figure}


\section{Sistema fotométrico UBV}
El sistema fotométrico UBV, es un sistema fotometrico de banda ancha para la clasificación estelar basado en el color del astro que observamos. 
\\

Se trata de una serie de filtros seleccionados de tal modo que la longuitud de onda principal que responda a ellos sea $364 \text{ nm}$ para U, $442 \text{ nm}$ para B, $540 \text{ nm}$ para V. El punto cero de B-V y U-B índice de color fue definido para ser cero para que las estrellas A0 V no se vieran afectadas por el enrojecimiento interestelar (fenómeno asociado a la extinción estelar).

\\

Es muy usual en Astronomía, y usaremos en algún ejercicio, el denominado índice de color B-V, que no es más que la diferencia entre la magnitud obtenida por el filtro B y la obtenida por el V: con este valor puede determinarse de modo muy aproximado la temperatura superficial de las estrellas y el tipo espectral al que corresponde.

\section{Atenuación atmosférica}
El fenómeno de atenuación atmosférica consiste en la absorción de luz estelar por parte de la atmósfera. Vamos a intentar obtener una expresión que nos permita conocer la magnitud real de una estrella, es decir fuera de la atmósfera, 
Partamos por definir un par de puntos de referencia entre la estrella y el observador. 
\\

Llamaremos $f_{\infty}$ al brillo original de la estrella y $f_{obs}$ al recibido por el observador. Además la cantidad de atmósfera que nos encontramos dependerá del ángulo de incidencia de la luz con el plano, y por ello podemos escribir en forma diferencial la distancia cenital con respecto a la estrella según $dl=dh \sec\theta$.
\\
Entonces el brillo escrito de forma diferencial vendrá dado por 
\begin{equation*}
    df=-\kappa \rho fdl
\end{equation*}
con $\kappa$ el coeficiente de extinción y $\rho$ la densidad de la atmósfera.
\\

Podemos usando esta expresión calcular el brillo total, 
\begin{equation*}
    \frac{df}{f}= - \kappa \rho dl = -\kappa \rho (\sec \theta) dh
\end{equation*}
e integrando para, 
\begin{equation*}
    \int_{f_{\infty}}^{f_{obs}}\frac{df}{f}=-\kappa \sec \theta \int_0^{H(atm)} \rho dh
\end{equation*}
si denotamos, $X_0$ a la integral en dh, podemos reescribir la expresión como: 
\begin{equation*}
    \ln\frac{f_{obs}}{f_{\infty}}=- \kappa X_0\sec \theta
\end{equation*}
y si despejamos de aqui, el $f_{obs}$ tenemos una relación entre ambos brillos, 
\begin{equation}
    f_{obs}=f_{\infty} e^{-\kappa X_0 \sec \theta}
\end{equation}
y si ahora relacionamos la magnitud fuera de la atmosfera, $m$ on la magnitud medida en la superficie $m´$, podemos escribir,
\begin{equation*}
    m´-m=-2,5 \log\frac{f_{obs}}{f_{\infty}}
\end{equation*}
y sustituyendo nos encontramos con que
\begin{align*}
    m= m-2,5 \log e^{-\kappa X_0 \sec \theta}=
    \\
    = m+(2,5\log e) \kappa X_0 \sec \theta
\end{align*}
y finalmente llegamos a la expresión, 
\begin{formula}
   \begin{equation*}
    m´=m+1,086 \kappa X_0 \sec \theta
\end{equation*} 
\end{formula}
por lo que tenemos una expresión que nos relaciona la magnitud que percibimos debido a la atenuación atmosférica con la magnitud original de la estrella

\section{Lineas de emisión y espectros estelares}
\begin{quotation}
``Jamás podremos saber de qué están hechas las estrellas'', por lo tanto la ciencia y la filosofía debían eliminar de sus objetivos ``algo que las enormes distancias nos impedirán saber jamás''.
\end{quotation}
\begin{flushright}
   \noindent Auguste Comte, \textit{Tratado de Filosofía Positiva}, 1830. 
\end{flushright}


Si hacemos pasar un haz de luz blanca por un prisma de vidrio, la luz se descompone en los colores del arcoíris. El espectroscopio (al igual que podemos ver en el la portada del disco de Pink Floyd, The Dark Side of The Moon) es el instrumento astronómico que descompone de igual modo la luz de cualquier astro. Si las estrellas se comportaran como cuerpos negros podríamos esperar un espectro continuo, en el que la temperatura de la estrella se
notaría en la intensidad de los colores, pero no.  
La realidad es más compleja: en la atmósfera estelar existen átomos de diferentes elementos químicos capaces de absorber determinadas longitudes de
onda de la radiación luminosa. El efecto son determinadas líneas oscuras en el espectro, debido a la pérdida de la energía absorbida. 

\\

La clasificación O-B-A-F-G-K-M recoge los tipos principales de espectros. Por supuesto, las estrellas reales tienen espectros intermedios, lo que se recoge añadiendo al
tipo fundamental un número entero entre 0 y 9. Así, una estrella de tipo B1 tiene un espectro bastante parecido a una estrella O9 y una estrella de tipo B9 se parece mucho a
una estrella A0. Nuestro Sol, por ejemplo es una estrella de tipo G.


\begin{figure}[h]
    \centering
    \includegraphics[width=1\linewidth]{espectros_4.png}
    \caption{Los siete principales tipos espectrales, OBAFGK y M se muestran en esta imagen, con el espectro de una estrella «O» encabezando la lista, al que siguen dos espectros para cada uno de los tipos espectrales restantes, correlacionados respectivamente con temperaturas cada vez más frías.}
    \label{fig:placeholder}
\end{figure}

\section{Efecto Doppler}
Este fenómeno describe el cambio de frecuencia aparente de una onda producido por el movimiento relativo de la fuente respecto a su observador. 

\subsection{Efecto Doppler no relativista}
En el caso del espectro visible de la radiación electromagnética, si el objeto se aleja, su luz se desplaza a longitudes de onda más largas, produciéndose un corrimiento hacia el rojo. Si el objeto se acerca, su luz presenta una longitud de onda más corta, desplazándose hacia el azul.
\\



En el siguiente desarrollo, el subíndice $r$ referencia al receptor y el subindice $e$ al emisor. Por ello tenemos; 
\begin{align*}
    t_{r1}=t_{e1}+ \frac{d}{c}
\end{align*}
para un instante de tiempo, y
\begin{equation*}
        t_{r2}=t_{e2} + \frac{(d+c \cdot\Delta t_e)}{c}
\end{equation*}
para un instante posterior. 
\\

Si tomamos la diferencia entre ellos, 
\begin{equation*}
    \Delta t _r = \Delta t _e (1+ \frac{v}{c})
\end{equation*}
Y identificamos a cada diferencia con su correspondiente periodo, tenemos entonces, 
\begin{equation*}
    T_r=T_e(1+\frac{v}{c}) 
\end{equation*}
Tomamos ahora ${T=\frac{1}{v}} $, podemos relacionar velocidades,
\begin{equation*}
    \frac{1}{v_r}=\frac{1}{v_e}(1+\frac{v}{c})
\end{equation*}
y usando la relación $\lambda=\frac{c}{v}$, podemos establecer una relación entre las longitudes de onda, 
\begin{equation*}
    \lambda_r=\lambda_e \big( 1 + \frac{v}{c}\big)
\end{equation*}

En Astrofísica, $z$ es un parámetro observacional que relaciona la longitud de onda de una línea espectral emitida por una fuente en movimiento (medida en su sistema propio o en el laboratorio) $\lambda_e$ y la longitud de onda registrada por un observador en la Tierra, $\lambda_r$:
\begin{equation}
    z=\frac{\lambda_r-\lambda_e}{\lambda_e}=\frac{\Delta\lambda}{\lambda_e} \begin{cases}
        z>0 \text{ \hspace{0,5cm} corrimiento al rojo (red shift)} 
        \\
        z<0 \text{ \hspace{0,5cm} corrimiento al azul (blue shift)}
    \end{cases}
\end{equation}
En función del efecto Doppler clásico,
\begin{formula}
    \begin{equation*}
    \lambda_r=\lambda_e\big( 1+\frac{v}{c} \big) \rightarrow z=\frac{\Delta \lambda}{\lambda_e}=\frac{v}{c}
\end{equation*}
\end{formula}


La medida de $z$ nos proporciona $v$ de forma que $v=zc$, pero de esta forma para $z>1 \rightarrow v>c$, y esto es imposible, ¿que podemos hacer para solucionar este problema?
\\

Para $z \leq 0,2$ podemos usar la aproximación clásica, para velocidades mayores de esta, vamos a abandonarla y a estudiar la formulación relativista del efecto Doppler. 

\subsection{Efecto Doppler relativista}
\begin{figure}[h]
    \centering
    \includegraphics[width=1\linewidth]{doppler_rel.png}
    \caption{Dos pulsos emitidos por un emisor en movimiento, a una distancia $d$, para un receptor en reposo}
    \label{fig:placeholder}
\end{figure}

Los tiempos $t_{r1}$, $t_{r2}$, $t_{e1}$ y $t_{r2}$ se han medido en el sistema de referencia del receptor, por esa razón, $\Delta t_e$ no es el intervalo de
tiempo transcurrido entre la emisión de ambos pulsos.
\\
La diferencia entre los tiempos de emisión de ambos pulsos en el sistema de referencia propio del emisor es de $\Delta \tau_e$, mientras que en el sistema de referencia del receptor es $\Delta t_e$, que es más largo, debido a la dilatación del tiempo, $\Delta t_e > \Delta \tau_e$. 
\\
En efecto, el factor 
\begin{equation}
    \gamma=\frac{1}{\sqrt{1-\big(\frac{v}{c}\big)^2}}>1
\end{equation}
Por lo tanto podemos reescribir las expresiones del efecto Doppler clásico usando este factor para emisor 
\begin{align*}
    \Delta t_e=\frac{\Delta \tau_e}{\sqrt{1- (v/c)^2}}=\gamma \Delta \tau_e
\end{align*}
y receptor respectivamente 
\begin{equation*}
        \Delta t_r = \Delta \tau_e \frac{1+v/c}{\sqrt{1-(v/c)^2}}
\end{equation*}
 
y el periodo se escribe como. 
\begin{equation*}
    T_r=T_e \frac{1+v/c}{\sqrt{1-(v/c)^2}}
\end{equation*}
Podemos escribir ahora el efecto Doppler relativista para las longitudes de onda: 
\begin{equation*}
    \lambda_r = \lambda_e \frac{1 + v/c}{\sqrt{1-(v/c)^2}} = \lambda_e \sqrt{\frac{1+v/c}{1-v/c}}  = \lambda_e \sqrt{\frac{c+v}{c-v}}
\end{equation*}
Interpretando el parámetro $z$ como consecuencia del efecto Doppler relativista: 
\begin{equation*}
    z=\frac{\Delta \lambda}{\lambda} = \frac{\lambda_r-\lambda_e}{\lambda_e} = \sqrt{\frac{1+v/c}{1-v/c}}-1=\sqrt{\frac{c+v}{c-v}}-1
\end{equation*}
Se puede calcular $\beta=v/c$, a partir de $z$:
\begin{equation*}
    \beta=\frac{v}{c}=\frac{(z+1)^2-1}{(z+1)^2+1}<1
\end{equation*}
De esta forma solucionamos las velocidades mayores a la de la luz.

\begin{nota}
    El corrimiento hacia el rojo cosmológico (el valor $z$ para las galaxias muy lejanas) no se debe al efecto Doppler, sino que es un consecuencia de la expansión del universo
\end{nota}






\newpage
\begin{figure}
    \centering
    \includegraphics[width=1\linewidth]{helice.png}
    \caption{Imagen de la Nebulosa de la Hélice realizada mediante espectroscopia infrarroja. Tomada por el Telescopio espacial Spitzer de la NASA en 2007.}
    \label{fig:placeholder}
\end{figure}
\newpage

\chapter{El tamaño y la distancia a las estrellas. Sistemas binarios}

\begin{flushleft}
\Large\textit{««En los últimos milenios hemos hecho los descubrimientos más asombrosos e inesperados sobre el Cosmos y el lugar que ocupamos en él; seguir el hilo de estas exploraciones es realmente estimulante. Nos recuerdan que los hombres han evolucionado para admirarse de las cosas, que comprender es una alegría, que el conocimiento es requisito esencial para la supervivencia. Creo que nuestro futuro depende del grado de comprensión que tengamos del Cosmos en el cual flotamos como una mota de polvo en el cielo de la mañana»»}\\[0.5em]
\large--- Carl Sagan
\end{flushleft}

\section{La distancia a la estrellas. Método de Paralaje}

El cambio de posición de
una estrella en el cielo
debido al movimiento anual
de la Tierra en torno al Sol
depende del paralaje del
astro: ángulo con el que se
vería el radio de la órbita
terrestre en torno al Sol desde el astro estudiado.
Observemos en la imagen
que ese ángulo es
exactamente igual que el de
la órbita aparente del astro
observado desde la Tierra.
Una vez conocido el
paralaje la distancia del
astro al Sol se determinaría
como un simple problema de Matemáticas: se trataría de calcular un cateto de un
triángulo rectángulo, conociendo un ángulo (el paralaje) y el lado opuesto 
(la distancia
de la Tierra al Sol)


\begin{figure}[h]
    \centering
        \includegraphics[width=0.5\linewidth]{paralaje1.png}
        \label{parsec}
        \caption{Representación del paralaje entre tomando como referencias dos meses del año y un objeto celeste}
    
\end{figure}
Si $p$ es el ángulo de paralaje y $d$ la distancia desde el sol a la estrella, sabemos que $\tan(p)=1  \text{UA} /d$, de manera que si despejamos a distancia y asumismo que $p$ es un ángulo lo suficientemente pequeño podemos despejar $d$ como, 
\begin{equation}
    d=\frac{1 \text{UA}}{p}
\end{equation}
escribiendo el ángulo en radianes. 
\\
Como las distancias a las estrellas son muy grandes, conviene hacer un cambio de unidades. Si nos fijamos en Fig. \ref{parsec} el paralaje del astro $P$ es de 1 segundo de arco.  A la distancia a la que se encuentra un astro con un paralaje de un segundo de arco se le denomina parsec, como vimos anteriormente en el capitulo 1. 
\\

Tras este recordatorio sobre el método de paralaje, vamos a estudiar como podemos conocer el radio y la masa de una estrella, para ello recuperaremos conceptos del capitulo 2 y los aplicaremos a sistemas binarios. 
\section{Sistemas binarios de estrellas}

\begin{definicion}
    Una estrella binaria es un sistema estelar compuesto de dos estrellas que orbitan mutuamente alrededor de un centro de masas común.
\end{definicion}
Estudios recientes sugieren que un elevado porcentaje de las estrellas son parte de sistemas de al menos dos astros. Los sistemas múltiples, que pueden ser ternarios, cuaternarios, o inclusive de cinco o más estrellas interactuando entre sí, suelen recibir también el nombre de estrellas binarias, como es el caso de Alfa Centauri A y B y Próxima Centauri. Nosotros vamos a clasificar estos sistemas dependiendo de su método de detección. 
\subsection{Estrellas binarias visuales}

Las estrellas binarias que se pueden distinguir visualmente con la ayuda de un telescopio, se llaman binarias visuales. Esta designación depende claramente de la potencia de resolución del telescopio usado, pero proporciona una clasificación útil. Esta clase incluye el espectacular ejemplo de Cygni-61 en el vecindario cercano del Sol.

\\

Las órbitas binarias pueden contribuir a la medición de las masas de diferentes clases de estrellas que aparecen en tales sistemas. A partir de la medición del periodo y el semieje mayor de la órbita de las estrellas binarias, se puede obtener la suma de las masas de las estrellas, si se conoce la distancia a la pareja. Además, si se puede medir la órbita de cada una de las estrellas, se pueden deducir las masas individuales.



\begin{figure}[h]
    \centering
    \includegraphics[width=1\linewidth]{cygni.png}
    \caption{Movimiento propio de 61 Cygni mostrando su posición con intervalos de un año.}
    \label{fig:placeholder}
\end{figure}

\begin{figure}[h]
    \centering
    \includegraphics[width=1\linewidth]{cygni61.png}
    \caption{Posición de 61 Cygni dentro de la constelación de Cygnus}
    \label{fig:placeholder}
\end{figure}


Vamos a deducir las expresiones relacionadas con estos sistemas que nos van a permitir estimar su masa.
\\

Partamos de la tercera ley de Kepler, introducida en el capítulo 2. 

\begin{equation}
    \frac{a^3}{T^2}=\frac{G(M_1+M_2)}{4 \pi^2}
    \label{tercerakep}
\end{equation}
Respecto del centro de masas, podemos escribir, 
\begin{equation*}
    \frac{a_1}{a_2}=\frac{M_2}{M_1} \hspace{1cm}a=a_1+a_2
\end{equation*}
despejando podemos llegar a 
\begin{equation*}
    M_2=\frac{a_1}{a_2}M_1 \rightarrow M_1 +M_2 = \Bigl(\frac{a_1}{a_2}+1\Bigl) M_1 = \frac{a_1+a_2}{a_2}M_1=\frac{a}{a_2}M_1
\end{equation*}

Sustitueyendo en \ref{tercerakep}
\begin{equation*}
    \frac{a^3}{T^2}=\frac{G}{4 \pi^2}\frac{a}{a_2}M_1\rightarrow\frac{a^2}{T^2}=\frac{G}{4\pi ^2} \frac{M_1}{a_2} 
\end{equation*}

De donde se llega a las siguientes expresiónes para cada masa
\begin{align*}
    M_1=\frac{4 \pi^2}{G} \frac{a^2a_2}{T^2}
    \\
     M_2=\frac{4 \pi^2}{G} \frac{a^2a_1}{T^2}
\end{align*}
de forma que conociendo el el periodo y la distancia al foco podemos conocer las masas de cada una de las estrellas.Si las masas son iguales podemos escribir: $M_1=M_2=M_*$, $a_1=a_2=a_*$ y $a=2a_*$ entonces, 
\begin{equation*}
    M_*=\frac{16\pi^2}{G}\frac{a_*^3}{T^2}
\end{equation*}

\begin{figure}[h]
    \centering
    \includegraphics[width=0.75\linewidth]{doble_doble.png}
    \caption{Épsilon Lyrae no es solo una estrella doble, sino una doble doble.}
    \label{Doble_Doble}
\end{figure}

Los sistemas triples de estrellas son dinámicamente inestables, salvo en el caso de que una
tercera estrella orbite a gran distancia alrededor de un sistema doble. En \ref{Doble_Doble} se representa un
sistema formado por dos binarias que se orbitan mutuamente.

\subsection{Binarias eclipsantes}
Se entiende por binaria eclipsante un sistema formado por dos estrellas cuyo plano orbital está orientado hacia la Tierra, de tal modo, que desde nuestra perspectiva, sufren eclipses y tránsitos mutuos. El eclipse o tránsito puede ser total o parcial. A diferencia de otros conjuntos binarios, las binarias eclipsantes no permiten distinguir sus estrellas unas de otras. Sin embargo, gracias a los cambios de luminosidad periódico en estos sistemas, se puede deducir cuando una estrella oculta a la otra. 
\\

Para estudiar este tipo de sistema debemos atender a la curva de luz, donde se representan los cambios periodicos de luminosidad del sistema, como podemos apreciar en la figura \ref{eclipsantes}. Buscaremos una expresión que nos permita calcular los radios de estas estrellas, si nos fijamos en la Fig. \ref{fig:eclipsantes}, podemos aplicar la siguiente relación trigonométrica entre los dos tiempos que se definen, $t_1$ cuando ambas estrellas son tangentes de forma interior, y $t_2$ cuando lo son de forma exterior. El tiempo de referencia $t_0$ es cuando ambas son totalmente concéntricas. Entonces tenemos que, 
\begin{equation}
    \begin{cases}
        \sin(\omega t_1)=\frac{R_1-R_2}{a}
        \\
        \sin(\omega t_2)=\frac{R_1+R_2}{a}
    \end{cases}
\end{equation}
sumando ambas expresiones obtenemos
\begin{equation}
    \frac{R_1}{a}=\frac{1}{2}(\sin(\omega t_1)+ \sin(\omega t_2))
\end{equation}
y si las restamos tenemos que, 
\begin{equation}
    \frac{R_2}{a}=\frac{1}{2}(\sin(\omega t_1)-\sin(\omega t_2))
\end{equation}
Ambas expresiones relacionan magnitudes observacionales (provinientes de la curva de luz) con el radio de cada una de las estrellas. (Ver Fig. \ref{fig:eclipsantes})

\begin{figure}[h]
    \centering
    \includegraphics[width=0.75\linewidth]{eclipsantes.png}
    \caption{Las observaciones de la curva de luz dan la magnitud total del sistema en función del tiempo. La periodicidad de la serie temporal se analiza para obtener la curva de luz en función de la fase.}
    \label{eclipsantes}
\end{figure}

\begin{figure}
    \centering
    \includegraphics[width=0.75\linewidth]{binarias_eclipsantes.png}
    \caption{El problema de las binarias eclipsantes puede simplificarse si empleamos la trigonometríca, vease, que se forma un tríangulo rectángulo tomando dos tiempos de la órbita de alguna de las estrellas}
    \label{fig:eclipsantes}
\end{figure}



\subsection{Binarias espectroscópicas}
Las estrellas binarias pueden ser detectadas a partir de la superposición de diferentes tipos de espectros (ver Fig. \ref{Binarias_espec}). Las modernas mediciones interferométricas, se han sumado a nuestra capacidad para distinguir y estudiar las estrellas binarias. Tenemos que aplicar el efecto Doppler a los desplazamientos al rojo y al azul que percibimos. Estudiemos primero el caso más sencillo, cuando la inclinación es de 90 grados, $i=90º$ (Ver Fig\ref{fig:90}). 
\\

Escribimos el efecto Doppler obtenido en el tema 3 y obtenemos una velocidad Doppler.
\begin{equation}
    v=\frac{d}{dt}(a\cos(\omega t)=-a \omega \sin(\omega t)
\end{equation}

El caso para $i \neq 90º$ volvemos a aplicar la expresión para calcular la velocidad Doppler (ver Fig. \ref{espec_no90}) , tenemos entonces que
\begin{equation*}
    v=\frac{d}{dt}(a\cos \omega t  \sin i ) = - \omega a  \sin (i) \sin( \omega t) 
\end{equation*}
consideraremos las dos masas iguales y las órbitas circulares, reescribiendo $\omega$ como $\omega=\frac{2\pi}{T}$ podemos escribir
\begin{equation*}
    v=-\frac{2 \pi}{T}a\sin (i) \sin( \omega t) = v_0 \sin \omega t 
\end{equation*}
donde hemos obtenido de la amplitud de la curva de velocidad radiales $v_0$ y del periodo $T$ de forma que tenemos
\begin{equation*}
    a \sin i = \frac{T}{2 \pi} |v_0|
\end{equation*}
podemos escribir entonces
\begin{equation*}
    M_*=\frac{16 \pi ^2}{G}\frac{a^3}{T^2} 
\end{equation*}
pero como solo conocemos $a\sin i $, entonces estamos límitados a calcular $M_* \sin^3i$. Resolvamos un problema a modo de ejemplo para revisar lo visto en esta sección 



\begin{figure}[h!]
    \centering
    \includegraphics[width=1\linewidth]{espectro_90.png}
    \caption{Caso más sencillo para aplicar el efecto Doppler donde la inclinación es $i=90º$}
    \label{fig:90}
\end{figure}

\begin{figure}
    \centering
    \includegraphics[width=1\linewidth]{binarias_esp_gen.png}
    \caption{Caso más general para el estudio de binarias espectrosc
    ópicas, donde la inclinación es distinta a $90º$}
    \label{espec_no90}
\end{figure}




\begin{figure}
    \centering
    \includegraphics[width=1\linewidth]{resumen_binarias.png}
    \caption{Resumen de las ecuaciones usadas para los sistemas binarios. Realizado por el profesor Antonio José Cuesta Vázquez}
    \label{fig:resumen_binarias}
\end{figure}
\begin{figure}
    \centering
    \includegraphics[width=1\linewidth]{binarias_espectroscopicas.png}
    \caption{Las estrellas binarias se pueden detectar a partir del desplazamiento al rojo del espectro de una estrella mientras que la otra se desplaza hacia el azul.}
    \label{Binarias_espec}
\end{figure}

\\

\begin{ejemplo}[Masa de un sistema binario]
Se tiene un sistema estelar binario situado a $10 \text{pc}$ en el que se observa que
la máxima separación angular de sus componentes es de $\theta_{max}=7''$ y que la mínima es de $\theta=1''$
. Su período orbital es de 100 años. Determine la masa de este sistema,
admitiendo que el plano orbital es normal a la línea de visión. 
\\

Vamos a pasar esta separación máxima y mínima a unidades astronómicas, usamos el factor 
\begin{formula}
    \begin{equation}
    1''=\frac{1 \text{UA}}{1\text{pc}}
\end{equation}
\end{formula}

lo que nos permite entonces escribir
\begin{align}
    7'' \rightarrow 7'' \cdot\frac{1º}{3600''} \cdot \frac{2 \pi}{360º} \cdot10 \text{pc} = 3,4 \cdot10^{-4} \text{ pc} \cdot\frac{206263 \text{ UA}}{1\text{pc}}=70 \text{UA}
    \\
        7'' \rightarrow 7'' \cdot\frac{1º}{3600''} \cdot \frac{2 \pi}{360º} \cdot10 \text{pc} = 4,9 \cdot10^{-5} \text{ pc} \cdot\frac{206263 \text{ UA}}{1\text{pc}}=10 \text{UA}
\end{align}
y además sabemos que $r_{max}+r_{min}=2a$, entonces podemos sacar un valor numérico para nuestro semieje $a=40 \text{ UA}$
\\

Usamos la tercera ley de Kepler para obtener la masa: 
\begin{equation*}
    \frac{a^3}{T^2}=G \cdot\frac{M_T}{4 \pi^2} \rightarrow M_T= 1,27 \cdot10^{31} \text{kg}
\end{equation*}
\end{ejemplo}

\section{Clasificación estelar. El diagrama de Hertzprung-Russel}
El diagrama de Hertzsprung-Russell es un gráfico de dispersión de estrellas que indica la relación entre las magnitudes absolutas o luminosidades de las estrellas en comparación con sus clasificaciones espectrales o las temperaturas efectivas. De forma más sencilla, en el gráfico se traza cada estrella para medir su brillo en comparación con su temperatura (color).


\begin{figure}[h]
    \centering
    \includegraphics[width=1\linewidth]{H-R.png}
    \caption{Diagrama de Hertzsprung-Russell.}
    \label{fig:placeholder}
\end{figure}
La mayoría de las estrellas ocupan la región del diagrama a lo largo de la línea llamada secuencia principal. Durante la etapa de su vida en la que las estrellas se encuentran en la línea de la secuencia principal, están fusionando hidrógeno en sus núcleos. La siguiente concentración de estrellas se encuentra en la rama horizontal (fusión de helio en el núcleo y combustión de hidrógeno en una envoltura que rodea al núcleo). Otra característica destacada es la brecha de Hertzsprung situada en la región entre el tipo espectral A5 y G0 y entre +1 y -3 magnitudes absolutas (es decir, entre la parte superior de la secuencia principal y las gigantes de la rama horizontal). Las estrellas variables RR Lyrae se encuentran a la izquierda de este hueco, en una sección del diagrama denominada franja de inestabilidad. Las variables cefeidas también se encuentran en la franja de inestabilidad, a mayor luminosidad. 
Estos diagramas se pueden ver en las Fig \ref{fig:diagrama}
\\

La misión Gaia de la ESA mostró varias características en el diagrama que no se conocían o que se sospechaba que existían. Encontró una brecha en la secuencia principal que aparece para Enanas M y que se explica con la transición de un núcleo parcialmente convectivo a un núcleo totalmente convectivo. 
\\

La mayoría de las estrellas están actualmente clasificadas bajo el sistema de \textbf{Morgan–Keenan} (MK), utilizando las letras O, B, A, F, G, K, y M, una secuencia que abarca desde las más calientes (tipo O) a las más frías (tipo M). Cada clase de letra se subdivide usando un dígito numérico, con el 0 para las estrellas más calientes y 9 para las más frías (por ejemplo: A8, A9, F0, F1 forman una secuencia de las más calientes a las más frías). La secuencia ha sido ampliada con clases de otras estrellas y objetos parecidos a estrellas que no encajan en el sistema clásico, tal como la clase D para enanas blancas y la clase C para estrellas de carbono



\begin{figure}
    \centering
    \includegraphics[width=1\linewidth]{DiagramaH-R.png}
    \caption{Diagrama HR con la franja de inestabilidad y sus componentes resaltados.}
    \label{fig:diagrama}
\end{figure}

\begin{figure}
    \centering
    \includegraphics[width=1\linewidth]{portada5.png}
    \caption{La brillante NGC 253 es una de las galaxias espirales más brillantes visibles, y también una de las más polvorientas. Algunos la llaman la Galaxia de la Moneda de Plata por su apariencia en pequeños telescopios, o simplemente la Galaxia del Escultor por su ubicación dentro de los límites de la constelación austral del Escultor. Descubierta en 1783 por la matemática y astrónoma Caroline Herschel, este universo-isla polvoriento se encuentra a tan solo 10 millones de años luz de distancia. Con unos 70 000 años luz de diámetro, NGC 253 es el miembro más grande del Grupo de Galaxias del Escultor, el más cercano a nuestro Grupo Local de Galaxias. Además de sus líneas de polvo espirales, zarcillos de polvo parecen elevarse desde su disco galáctico, entrelazado con cúmulos estelares jóvenes y regiones de formación estelar en este colorido retrato galáctico. El alto contenido de polvo acompaña a la frenética formación estelar, lo que le ha valido a NGC 253 la designación de galaxia con brotes de formación estelar. También se sabe que NGC 253 es una potente fuente de rayos X y rayos gamma de alta energía, probablemente debido a los agujeros negros cerca del centro de la galaxia.
}
    \label{fig:placeholder}
\end{figure}

\chapter{Estructura y evolución estelar}
\begin{flushleft}
\Large\textit{«Es razonable esperar que en un futuro no muy lejano seamos capaces de comprender algo tan simple como una estrella».
[It is reasonable to hope that in the not too distant future we shall be competent to understand so simple a thing as a star]}\\[0.5em]
\large--- Arthur Eddington
\end{flushleft}


Las estrellas no duran para siempre. Al igual que las personas, nacen, sufren cambios a lo largo de su vida y mueren, y tambien como nosotros, no todas ellas viven el mismo tiempo. El parámetro más relevante en la vida y formación de una estrella es su masa. Es esta masa inicial la que comienza a fusionar hidrógeno en su núcleo a un ritmo tal que se produce un equilibrio con la
propia gravedad de la Estrella. Fusionar hidrógeno a este ritmo evita que la estrella termine
colapsando sobre sí misma. 
\\

Cecilia Payne-Gaposchkin concluyo en su tesis doctoral, presentada en 1925 que las estrellas estaban formadas principalmente por hidrógeno y helio. Durante años este resultado fue descartado por el eminente astrónomo Henry Norris Russell, quien disuado a Payne de publicar este resultado. Años más tarde, otros cientificos confirmaron este resultado.
\\

Payne utilizo las leyes de la física cuántica para estudiar los espectros de emisión del Sol. 
%Aqui podemos añadir una imagen de estos espectros de emisión
\\

\section{Ecuaciones del interior estelar}

\subsection{Ecuación de equilibrio hidrostático}

El peso de una capa esférica de la estrella está compensado por la diferencia de presiones entre la parte interna y externa de dicha capa. 
En cada capa, el peso por unidad de volumen se equilibra con el gradiente de presiones. 
\begin{figure}[h]
    \centering
    \includegraphics[width=0.5\linewidth]{eq_hidrostática_b.png}
    \caption{La fusión del Hidrógeno emite energía que contrarresta a su propia gravedad.}
    \label{fig:placeholder}
\end{figure}
\\
Esta expresión se obtiene de imponer la condición de equlibrio entre la fuerza peso de un diferencial de volumen (hacia dentro) y de la diferencia neta de las fuerzas de presión (hacia fuera). Es la versión diferencial de la ecuación: 
\begin{equation*}
    P=\rho gh
\end{equation*}
siendo $g$ la intensidad del campo gravitatorio en esa capa 
\begin{equation*}
    g=\frac{GM_r}{r^2}
\end{equation*}
\begin{formula}[Ecuación del equilibrio hidrostático]
\[
\frac{dP}{dr}=-\rho(r)\frac{GM_r}{r^2}
\]
\end{formula}

Cuya solución para una función $\rho(r)$ dada (y para $M_r$ dada por la ecuación de conservación de la masa) nos dará el perfil radial de presiones $P(r)$ que ha de cumplir con las condiciones de contorno: 
\begin{align*}
    P(r=0)=P_0
    \\
    P(r=R) \approx 0
\end{align*}
donde el subíndice 0 significa central

\subsection{Ecuación de continuidad de la masa}
La contribución de una
capa esférica a la masa
total de la estrella
dependerá de la densidad
que haya a esa distancia $r$. Se obtiene de la definición de densidad local, que al no ser constante hay que calcularla en un elemento de volumen: 
\begin{equation*}
    \rho(r)\equiv \frac{dM_r}{dV}
\end{equation*}
Por ello se obtiene la siguiente expresión: 
\begin{formula}
    \begin{equation*}
        \frac{dM_r}{dr}=\rho(r)4\pi r^2
    \end{equation*}
\end{formula}
Cuya solución para una función $\rho(r)$ dada nos dará el perfil radial de masa $M_r(r)$ que ha de cumplir las condiciones de contorno:
\begin{align*}
    M_r(r=0)=0
    \\
    M_r(r=R)=M
\end{align*}
\subsection{Ecuación de balance de energía (Ecuación de la luminosidad)}
La contribución de una capa esférica a la luminosidad total de la estrella dependerá del valor de $\varepsilon$ y de la densidad que haya a esa distancia $r$. 
\\ 
\\
Se obtiene de la definición de $\varepsilon$ que es la luminosidad por unidad de masa: 
\begin{equation*}
    \varepsilon \equiv \frac{dL_r}{dM_r}
\end{equation*}

El valor de $\varepsilon$ se anula fuera del núcleo estelar

\begin{formula}
    \begin{equation*}
        \frac{dL_r}{dr}=\varepsilon \rho(r)4\pi r^2
    \end{equation*}
\end{formula}
Cuya solución para una función $\rho(r)$ dada nos dará el perfil radial de luminosidad $L_r$ que ha de cumplir las condiciones de contorno: 
\begin{align*}
    L_r(r=0)=0
    \\
    L_r(r=R)=L
\end{align*}
\subsection{Ecuación de transporte radiativo (Ecuación de la Temperatura)}
El cambio de temperatura de una capa esférica a la siguiente, el enfriamiento, dependerá del valor de $\kappa$ y de la densidad que haya a esa distancia $r$, así como del flujo de radiaicón que llega a dicha capa. 
\\
Es decir, no es más que: 
\begin{equation}
    dF=-\kappa F\rho \hspace{3pt}dr   
    \label{scattering}
\end{equation}
recordando que el flujo es proporcional a la temperatura a la cuarta. 
\\
La diferencia de temperatura entre dos capas consecutivas se debe a que el flujo que sale no es el mismo que el que entra, ya que $\kappa=0$.
La ecuación: 
\begin{formula}
    \begin{equation*}
        \frac{dT^4}{dr}=-\rho(r)\frac{\frac{\kappa}{4\pi \sigma}L_r(r)}{r^2}
    \end{equation*}
\end{formula}
Cuya solución dada para una función $\rho(r)$ dada y la $L_r(r)$ de a ecuación de balance de energía nos dará el perfil radial de temperaturas $T(r)$ que ha de cumplir las condiciones de contorno: 
\begin{align*}
    T(r=0)=T_0
    \\
    T(r=R)=T_{eff} \approx0
\end{align*}
Esta expresion se obtiene de igualar la versión diferencial de la ley de Stefan-Blotzamann, 
\begin{equation*}
    dF=\sigma dT^4
\end{equation*}
y la ecuación de la opacidad (scattering) (\ref{scattering}) y aplicando que el flujo F que llega de dicha capa está relacionado con la luminosidad generada dentro de la misma: 
\begin{equation*}
    F=\frac{L_r}{4\pi r^2}
\end{equation*}
\subsection{Ecuación de estado}

Las variables intensivas
dentro de una estrella
(presión, densidad, y
temperatura) están
relacionadas entre sí como
un gas ideal. 
\\
Esto nos dice que hay
equilibrio termodinámico
local: se puede definir una
presión, una densidad, y
una temperatura, pero sus
valores no son uniformes
en toda la estrella, sino
que su valor es distinto en
cada capa esférica.
\\
La expresión
\begin{formula}
    \begin{equation*}
        P(r)=\frac{k_B}{m}\rho(r)T(r)
    \end{equation*}
\end{formula}
no es una ecuación diferencial, simplemente nos dice que la solución de la ecuación del equilibrio hidrostatico y de transporte radiativo han de estar relacionadas entre sí mediante la densiadad que haya a cada valor de r.
\\
Esta expreisón se obtiene de la ecuación de los gases ideales pero expresada en función de la densidad de partículas $n$: 
\begin{equation*}
    P=nk_BT
\end{equation*}
donde la densidad de masa se puede escribir como $\rho=mn$ siendo m la masa promedio de una partícula del gas. 

\vspace{10mm}
En la parte convectiva de la estrella hay que reemplazar la ecuación de transporte radiativo por la ecuación de transporte convectivo: 

\subsection{Ecuación de transporte convectivo (Ecuación de la temperatura}

El cambio de temperatura
de una capa esférica a la
siguiente dependerá del
gradiente de presiones
en dicha capa, que viene
dado por la condición de
equilibrio hidrostático. 
\begin{equation*}
    \frac{T(r)}{P(r)}\frac{dP}{dr}=\frac{T(r)}{\frac{k_B}{m}\rho(r)T(r)}\Bigl( -\rho(r)\frac{GM_r(r)}{r^2} \Bigl)=-\frac{m}{k_B}\frac{GM_r(r)}{r^2}
\end{equation*}
Es decir el enfriamiento al aumentar la distancia al centro se debe a que el gas se expande de forma isoentrópica.
\\
La expresión aquí encontrada permite interpretar que el cambio en la energía cinétrica media de una partícula del gas es debido al cambio en su energía potencial: 
\begin{equation*}
    k_BdT=-\frac{GM_r(r)m}{r^2}dr
\end{equation*}

Esta ecuación se obtiene de usar la ecuación de los gases ideales junto con la condición de que se trata de un proceso isoentrópico: 
\begin{equation*}
    PV^\gamma \equiv \text{cte}
\end{equation*}
siendo $\gamma$, \begin{equation*}
    \gamma=\frac{c_p}{c_v} 
\end{equation*}
el índice adiabático, obteniendo una relación entre $P$ y $T$. 
\\
Tomando diferenciales y dividiendo por la ecuación sin diferenciar, se obtiene cómo están relacionados los perfiles radiales de presión y temperatura.

\begin{formula}
\begin{equation*}
    \frac{dT}{dr}=\Big(1-\frac{1}{\gamma} \Big)\frac{T}{P}\frac{dP}{dr}
\end{equation*}
\end{formula}
Cuya solución usando la función $P(r)$ dada por la solución de la ecuación de equilibrío hidrostático nos dara le perfil radial de temperaturas $T(r)$, que ha de cumplir la condiciones de contorno: 
\begin{align*}
    T(r=0)=T_0
    \\
    T(r=R)=T_{eff} \approx0
\end{align*}
\subsubsection{Transporte convectivo}
Consideremos ahora, una burbuja de materia estelar que asciende una distancia $\Delta r$. 
Las presiones se equilibran rapidamente con las del entorno pero no sus densidades: 
\begin{itemize}
    \item si $p^*>p$ la burbuja vuelve a caer, y tenemos una situación de equilibrio. 
    \item si $p^*<p$ la burbuja continua ascendiendo. Se da una situación de convección debido a al diferencia de temperaturas. 
    \\
\end{itemize}
 El ascenso de la burbuja es muy rápido y no tiene tiempo de intercambiar calor con el entorno. La burbuja sufre un proceso adiabático. 
La convección aparece si el gradiente de temperatura del entorno es ligeramente mayor que el gradiente adiabático, de modo que la burbuja esta más caliente que el entorno y tiene una menor densidad. 
Por lo que tenemos que, para un proceso de convección, 
\begin{equation*}
    |\frac{dT}{dr}|_{\text{estrella}} >|\frac{dT}{dr}|_{\text{adiab}}
\end{equation*}
Calculamos el gradiente de temperatura adibático, para ello partimos de  
\begin{equation*}
    PV^{\gamma}=\text{cte.}
\end{equation*}
pero aplicando la ley de los gases ideales llegamos a, 
\begin{equation*}
    P\big(\frac{NRT}{P}\big)^{\gamma} = \text{cte}
\end{equation*}
y de esta relación llegamos a, 
\begin{equation*}
    P^{1-\gamma}T^{\gamma}= \text{cte} \rightarrow P^{\frac{1-\gamma}{\gamma}}=\text{cte}
\end{equation*}
Tomando ahora logaritmos, 
\begin{equation*}
    \frac{1-\gamma}{\gamma}\ln P+ \ln T
=\text{cte}
\end{equation*}
y ahora podemos tomar derivadas
\begin{equation*}
\big( 1-\frac{1}{\gamma}\big)\frac{dP}{P}+\frac{dT}{T}=0
\end{equation*} \
\\
y llegamos a esta expresión,

\begin{formula}
    \begin{equation*}
    \frac{dT}{dr}\Big|_{\text{adiab}} = \left(1-\frac{1}{\gamma}\right)\frac{T}{P}\frac{dP}{dr} \le \frac{dT}{dr}\Big|_{\text{estrella}}
\end{equation*}
\end{formula}

En este punto, y para el desarrollo que continuaremos haciendo más adelante, vamos a introducir la siguiente expresión para conocer la masa media de las partículas.

\begin{formula}
   \begin{equation*}
    \overline{m}=\frac{m_p}{2}
\end{equation*}

\end{formula}


\subsection{Integración en un solo paso}
Encontrar la solución de una ecuación diferencial no siempre es sencillo, en esta sección, vamos a estudiar un método de aproximación para resolver las ecuaciones que rigen el interior estelar. 

\\ 

Este método sirve para relacionar el valor de las incógnitas en el centro $(r=0)$ y en el borde de la estrella $(r=R)$. Distinguimos ahora entre las variables que se anulan en el centro y las que se anulan en el borde de la estrella, para ello hemos confeccionado la siguiente tabla: 


    \begin{table}[h]
    \centering
    \begin{tabular}{|l|l|}
    \hline
    {\color[HTML]{3166FF} \textbf{Magnitud en el centro de la estrella $r=0$}} & {\color[HTML]{3166FF} \textbf{Magnitud en el borde de la estrella $r=R$}} \\ \hline
    $M_r(r=0)=0$                                                               & $M_r(r=R)=M$                                                              \\ \hline
    $L_r(r=0)=0$                                                               & $L_r(r=R)=L$                                                              \\ \hline
    $P_r(r=0)=P_0$                                                             & $P_r(r=R)=0$                                                              \\ \hline
        $\rho (r=0)= \rho_0$                                                       & $\rho (r=R)=0$ (es despreciable)                                          \\ \hline
    $T_r(r=0)=T_0$                                                             & $T_r(r=R)=0$                                                              \\ \hline
\end{tabular}
\caption{Relación de magnitudes en el núcleo y en el borde de una estrella}
\end{table}


El método de integración en un paso consiste en sustituir, en las ecuaciones del interior estelar las diferenciales por el incremento entre $r=R$ y $r=0$, $dx=x(r=R)-x(r=0)$ y las funciones incognita que no tengan diferenciales por su valor medio entre $r=R$ y $r=0$, es decir;
\begin{equation*}
    X=\frac{x(r=R)-x(r=0)}{2}
\end{equation*}

Veamos un ejemplo y que conclusiones podemos sacar de este método, 
\begin{ejemplo}
    Para la presión usaremos la siguiente expresión
    \begin{equation*}
        \frac{dP}{dr}=-\rho(r) \frac{GM_r}{r^2}
    \end{equation*}
y ahora aplicamos nuestro método
\begin{equation*}
    0-P_0=-G\frac{(\frac{(M+0)}{2})(\frac{(0+ \rho_0)}{2})}{(\frac{R+0}{2})^2}(R-0)
\end{equation*}
simplificamos y llegamos a este resultado: 
\begin{equation}
P_0=G\frac{M \rho_0}{R}    
\end{equation}
\label{Presion}
\end{ejemplo}

Se deja como ejercicio el desarrollo del resto de expresiones usando este método.
Las soluciones de las ecuaciones diferenciales son, para la masa:
\begin{equation}
M=\frac{\pi}{2}\rho_0R^3
\label{masa}
\end{equation}
para la ecuación del balance de energía: 
\begin{equation}
    L=\frac{\pi}{2}\rho_0 R^3
\end{equation}
y de la ecuación de transporte radiativo obtenemos: 
\begin{equation}
    L=\frac{2\pi\sigma}{\kappa\rho_0}RT^4_0
    \label{transporte}
\end{equation}
la primera conclusión que podemos obtener es una expresión particular para la  ecuación de estado: 
\begin{equation}
    P_0=\frac{2\rho_0}{m_p}k_bT_0
    \label{estado}
\end{equation}

Con estas expresiones obtenidas de resolver las ecuaciones diferenciales anteriores mediante aproximaciones, podemos llegar a una relación entre la masa y la luminosidad. Partimos de \ref{Presion} y \ref{masa}, despejando $p_0$ de \ref{masa}, tenemos que \begin{equation*}
    \rho_0=\frac{2M}{\pi R^3}
\end{equation*}
y si ahora sustituimos en \ref{Presion} tenemos que, 
\begin{equation*}
    P_0=\frac{2GM}{\pi R^4}
\end{equation*}
y ahora añadimos la ecuación de estado \ref{estado}, que nos permite escribir la siguiente relación entre los valores de $P_0$ y $\rho_0$ que hemos despejado anteriormente con $T_0$, de forma que: 
\begin{equation*}
    T_0=\frac{GMmp}{2Rk_b}
\end{equation*}
si tomamos la ecuación \ref{transporte} podemos sustituir el valor de $T_0$ obtenido y tenemos que, 
\begin{equation}
    L=\frac{\pi^2\sigma}{\kappa}\left( \frac{Gm_P}{2k_b} \right)^4M^3
\end{equation}
hemos dejado una variable en función de la otra mediante una constante universal. Por lo tanto la relación entre la masa y la luminosidad es: 
\begin{formula}
    \begin{equation}
        \label{relación_masa_luminosidad}
        L\propto M
    \end{equation}
\end{formula}
\\
Podemos escribir una relación con los valores del Sol que son conocidos, y por lo tanto tenemos que: 
\begin{equation}
    \frac{L_*}{L_{\odot}}= \left(\frac{M_*}{M_{\odot}} \right)
\end{equation}

El resto de relaciones que se detallan a continuación son aproximadas:
\\ 
\begin{enumerate}[label=(\roman*)]
    \item Podemos escribir la siguiente relación radio-masa, considerando $T_0$ como una constante universal, ya que es casi la misma para todas las estrellas de la secuencia principal. 
    \begin{equation*}
    \frac{M}{R}=\frac{2T_0k_b}{Gm_p} \approx\text{cte} 
    \end{equation*}
    y la relación radio masa es: 
    \begin{equation*}
        R \propto M^3
    \end{equation*}
    \item la relación entre la densidad (central) y la masa, tomamos 
    \begin{equation*}
        \rho_0=\frac{2M}{\pi R^3}
    \end{equation*}
    y la relación radio-masa, y tenemos
    \begin{formula}
     \begin{equation*}
        \rho_0 \propto M^{-2}
    \end{equation*}
    \end{formula}
    \item La relación entre la producción de energía (emisividad) de una estrella y la masa, teniedo en cuenta la relaciones de radio y densidad anteriores: 
    \begin{equation}
        \varepsilon_0=\frac{4}{\pi} \frac{L}{\rho_0R^3} \propto \frac{1}{M^{-2}}\frac{M^3}{M^3} 
    \end{equation}
    Por lo que llegamos a la expresión
    \begin{formula}
        \begin{equation*}
           \varepsilon_0 \propto M^2
        \end{equation*}
    \end{formula}
    \item Tambien podemos relacionar la presión central con la masa, para ello: 
    \begin{equation}
        P_0=\frac{2}{\pi}\frac{GM^2}{R^4} \propto\frac{M^2}{M^4}
    \end{equation}
    simplificando términos llegamos a la siguiente relación:
    \begin{formula}
        \begin{equation*}
         P_0 \propto M^{-2}
        \end{equation*}
    \end{formula}
\end{enumerate}
Usando estas relaciones podemos conocer el tiempo de vida (nuclear) de una estrella en relación con su masa, tenemos que: 
\begin{equation*}
    t_{nuc} \propto\frac{M}{L} \propto \frac{M}{M^3} \propto \frac{1}{M^2}
\end{equation*}
entones llegamos a esta relación: 
\begin{formula}
    \begin{equation*}
        t_{nuc} \propto M^{-2}
    \end{equation*}
\end{formula}

Es decir, a pesar de que las estrellas más masivas tengan más combustible nuclear, al tener una mayor luminosidad, este se agota antes. 

\section{Reacciones nucleares en el interior de las estrellas}
A pesar de la dificultad técnica que supone para la humanidad, la fusión nuclear es la fuente de energía más común en el universo. 
En las estrellas se llevan a cabo dos clases de reacciones nucleares, por un lado cadenas protón-protón, donde 4 núcleos de hidrogeno se fusionan para obtener dos núcleos de helio-3, que posteriormente reaccionan entre si, para generar un núcleo de helio-4 y dos núcleos de hidrogeno. Durante el proceso, se obtiene neutrinos, veamos las reacciones: 
\begin{equation*}
    {}^{1}_{1}\mathrm{H} + {}^{1}_{1}\mathrm{H}
\;\longrightarrow\; {}^{2}_{1}\mathrm{H} + e^{+} + \nu_{e}
\end{equation*}
tras la fusión de lso dos protones, el deuterio, acepta otro protón transmutando en un núcleo de helio-3
\begin{equation*}
    {}^{2}_{1}\mathrm{H} + {}^{1}_{1}\mathrm{H}
\;\longrightarrow\;
{}^{3}_{2}\mathrm{He} + \gamma
\end{equation*}
ahora, dos nucleos de helio-3, se fusionan obteniendo un núcleo de helio-4 y dos protones extras, que continuan la reacción en cadena 
\begin{equation*}
    {}^{3}_{2}\mathrm{He} + {}^{3}_{2}\mathrm{He}
\;\longrightarrow\;
{}^{4}_{2}\mathrm{He} + 2\,{}^{1}_{1}\mathrm{H}
\end{equation*}
Esta cadena de reacciones nucleares se inicia cuando $T_0 \approx 10^7 \text{K}$
\\
\begin{figure}
    \centering
    \includegraphics[width=0.5\linewidth]{cadena_protón_protón.png}
    \caption{Cadena protón-protón}
    \label{fig:placeholder}
\end{figure}

\\

A temperaturas más altas se produce el ciclo del carbono, relevante en estrellas de mayor masa. El proceso se ve detallado en la imagen \ref{ciclo_del_carbono}
\begin{figure}
    \centering
    \includegraphics[width=0.5\linewidth]{ciclo_CNO.png}
    \caption{Ciclo CNO}
    \label{ciclo_del_carbono}
\end{figure}

Las energías obtenidas por cada uno de estos procesos son, para la cadena protón-protón, $E_{pp}=26,2 \text{ MeV}$ y para el ciclo CNO, $E_{CNO}=25,0 \text{ MeV}$, obteniendo un rendimiento de  
\begin{align*}
    \eta=\frac{E_{pp}}{4m_pc^2}=0,007 
\end{align*}
y de
\begin{align*}
   \eta=\frac{E_{CNO}}{4m_pc^2}=0,0067
\end{align*}
respectivamente. Esta diferencia entre un proceso y una radica en la energía que se escapa por los neutrinos que se extraen de cada proceso.

\section{Contracción gravitatoria de una estrella}
La energía potencial que adquiera la masa $dM_r$ al caer desde el infinito hasta una distancia $r$ del centro de la estrella vale,
\begin{equation*}
    d\Omega=-G\frac{M_r}{r}dM_r
\end{equation*}
integrando se obtiene la energía potencial total de estrella: 
\begin{equation}
    \Omega=-G\int_0^M\frac{M_r}{r}dM_r
\end{equation}
Podemos obtener un valor aproximación sacando fuera del signo integral un valor medio de $r$, que aproximaremos por $\frac{R}{2}$, 
\begin{equation*}
    \Omega \approx-G\frac{1}{r}\int_0^M M_r dM_r=-G\frac{2}{R} \int_0^MM_rdM_r=-G\frac{2}{R}\frac{1}{2}M^2
\end{equation*}
llegamos entonces a la siguiente expresión; 
\begin{formula}
\begin{equation}
\Omega \approx-G\frac{M^2}{R}  
\end{equation}  
\end{formula}


\section{Enanas Blancas}
Las enanas blancas (ver Fig. \ref{enana} y \ref{fig:enanab}) son un remanente estelar que se genera cuando una estrella de masa menor que 10 masas solares ha agotado su combustible nuclear. El físico Stephen Hawking, en el glosario de su obra \textit{Historia del tiempo} define la enana blanca de a siguiente manera: 
\begin{quote}
    Estrella fría estable, mantenida por la repulsión debida al principio de exclusión entre electrones
\end{quote}
Las enanas blancas están compuestas por átomos en estado de plasma; como en su núcleo ya no se producen reacciones termonucleares, la estrella no tiene ninguna fuente de energía que equilibre el colapso gravitatorio, por lo que la enana blanca se va comprimiendo sobre sí misma debido a su propio peso. 

Para estudiar a las enanas blancas vamos a considerar dos modelos, uno no relativista y otro relativista. 
\subsection{Enanas blancas no relativistas}
De las ecuaciones del interior estelar estudiadas anteriormente solo necesitamos tres para caracterizar a las enanas blancas. Como ecuación de estado, usaremos la siguiente que relaciona presión y densidad: 
\begin{formula}
    \begin{equation}
        P=K_1 \rho^{5/3}
        \label{estado:enana}
    \end{equation}
siendo $K_1 \approx3 \times10^6 \text{kg}^{-\frac{2}{3}} \text{m}^4\text{s}^{-2}$
\end{formula}
El transporte de energía es por conducción, muy eficaz por lo que la temperatura de la estrella es uniforme. Tomamos ahora la ecuación del equilibrio hidrostático
\begin{equation}
    P_0=\frac{\rho_0GM}{R}
    \label{equilibrio:enana}
\end{equation}
y la ecuación de continuidad
\begin{equation}
    M=\frac{\pi}{2} \rho_0 R^3
    \label{continuidad:enana}
\end{equation}
Podemos sacar ciertas conclusiones del estudio de estas ecuaciones. 
\begin{itemize}
    \item Si partimos de la ecuación \ref{equilibrio:enana} y de \ref{estado:enana}, podemos llegar a
    \begin{equation*}
        \frac{\rho_0GM}{R}=K_1 \rho_0^\frac{5}{3}
    \end{equation*}
    de donde despejamos $\rho_0$ llegamos a 
    \begin{equation*}
        \rho_0=\Bigl(\frac{GM}{K_1} \Bigl)^{3/2}R^{-3/2}
    \end{equation*}
    \item Si sustituimos en \ref{continuidad:enana} llegamos a 
    \begin{equation*}
        M=\frac{\pi}{2} \Bigl( \frac{GM}{K_1} \Bigl)^{\frac{3}{2}}R^{-\frac{3}{2}}R^3 \rightarrow M= \Bigl( \frac{2}{\pi}\Bigl)^2 \Bigl( \frac{K_1}{G}\Bigl)^3R^{-3}
    \end{equation*}
\end{itemize}
Por lo tanto notemos que, la masa se relaciona con el radio como 
\begin{equation}
    \label{rel_masa_radio_enana}
    M \propto R^{-3} \rightarrow R \propto M^{-1/3}
\end{equation}
Es decir el radio de la estrella disminuye conforme la masa aumenta. 
\begin{ejemplo}
    El radio tipio de una estrella de una masa solar será 
    \begin{equation*}
        R=\Bigl(\frac{2}{\pi} \Bigl)^{2/3} \frac{K_1}{G}M^{-1/3}=2600 \text{km}
    \end{equation*}
y su densidad media: \begin{equation*}
    \overline{\rho}=\frac{3M}{4\pi R^3}=2,6 \times10 ^9 \text{km/m}^3 =2,6 \times10^6 \text{g/cm}^3 
\end{equation*}
la densidad por lo tanto cumple la relación 
\begin{equation*}
    \rho_0 \propto R^{-6} \rightarrow \rho_0 \propto M^2
\end{equation*}
\end{ejemplo}

\subsection{Enanas blancas relativistas}
Las enanas blancas son objetos muy densos esto hace que los electrones viajen en su interior a velocidades cercanas a la de la luz, cuando una partícula viaja a estas velocidades debemos abandonar la aproximación clásica y usar un modelo relativista para la ecuación de estado
\\

En nuestro caso vamos a obviar las posibles etapas intermedias de las estrellas y acogernos al caso ultrarelativista. Por lo tanto, ahora la ecuación de estado es: 
\begin{formula}
    \begin{equation}
        P_0=K_2\rho_0^
        {4/3}
        \label{estado_ultra}
    \end{equation}
\end{formula}
Partamos de \ref{equilibrio:enana} y de \ref{estado_ultra} igualandolas llegamos a
\begin{equation*}
    \frac{\rho_0 GM}{R}=K_2 \rho_0^{4/3} \rightarrow\rho_0= \Bigl( \frac{GM}{K_2R}\Bigl)^3
\end{equation*}
si ahora tomamos la ecuación de continuidad \ref{continuidad:enana} y sustituimos este valor 
\begin{equation*}
    M=\frac{\pi}{2}\Bigl( \frac{GM}{K_2R}\Bigl)^3R^3 \rightarrow M^2=\frac{2}{\pi}\Bigl( \frac{K_2}{G}\Bigl)^3\equiv\textbf{cte}
\end{equation*}
Es decir, la masa de una enana blanca relativista no depende de su radio, o mejor dicho, el radio no queda determinado por su masa. Lo que acabamos de obtener es el límite máximo para la masa de una enana blanca, conocido como masas de Chandrasekhar y cuyo valor es: 
\begin{equation*}
    M_{CH}=\Bigl(\frac{2}{\pi}\Bigl)^{1/2}\Bigl(\frac{K_2}{G}\Bigl)^{3/2}
\end{equation*}
\begin{formula}
    \begin{equation}
        M_{CH}=1,44 M_{\odot}
    \end{equation}
\end{formula}

\vspace{1cm}

\begin{figure}[h]
    \centering
    \includegraphics[width=1\linewidth]{enanablanca.png}
    \caption{Comparación de tamaños entre la enana blanca IK Pegasi B (centro abajo), su compañera de clase espectral A IK Pegasi A (izquierda) y el Sol (derecha). Esta enana blanca tiene una temperatura en la superficie de 35 500 K.}
    \label{fig:enanab}
\end{figure}

Cuando una enana blanca supera el límite de Chandrasekhar tras acumular masa, normalmente mediante un proceso de acreción, acaba colapsando y provoca una supernova del tipo Ia, la cual es una herramienta fundamental en cosmología. 
Hemos estudiado en este tema que la luminosidad esta relacionada con la masa de una estrella, entonces todas las supernovas de este tipo tienen el mismo brillo y aplicando las expresiones del capitulo 3, podemos estimar la distancia a dicha supernova. Estrellas más masivas (aproximadamente más de ocho masas solares) no forman tras agotar su material nuclear enanas blancas, gran parte de ellas forman lo que conocemos como estrellas de neutrones. Conocemos tambien una masa límite para este tipo de estrellas, el límite de Tolman-Oppenheimer-Volkoff: 
\begin{formula}
    \begin{equation}
        M_{TOV}=2,55M_{\odot}
    \end{equation}
\end{formula}


\begin{figure}
    \centering
    \includegraphics[width=1\linewidth]{e_b.png}
    \caption{Nebulosa planetaria M27 o nebulosa Haltera. La pequeña estrella central es una enana blanca cuyo radio se estima en 0.055 radios solares aproximadamente. Esto la convierte en la enana blanca más grande conocida.
Créditos: Red de Telescopios Robóticos del Centro de Astrobiología (CSIC-INTA).}
    \label{enana}
\end{figure}

\chapter{Cosmología Newtoniana}

\begin{flushleft}
\Large\textit{«Hay dos cosas infinitas: el Universo y la estupidez humana, y de la primera no estoy seguro»}\\[0.5em]
\large--- Albert Einstein
\end{flushleft}

La cosmología es la rama de la física que estudia el universo como un objeto físico a diferencia de la Astrofísica que estudia los objetos contenidos en este. 
No es imposible hablar de cosmología sin hacerlo de la expansión de universo y de nuestros chivatos, las supernovas tipo Ia. Todas las supernovas tipo Ia explotan con la misma masa y por lo tanto tienen la misma luminosidad (ver relación en el capitulo 5) lo que nos permite definir una medida para la luminosidad como ya vimos al final del tema anterior. 

\section{Expansión del universo. La Ley de Hubble}
Se ha observado que las distancias entre las grandes estructuras del universo (los cúmulos y supercúmulos de galaxias) se incrementan de manera progresiva. Este hecho observacional se denomina expansión del universo y fue descubierto por Edwin Powell Hubble y Milton Lasell Humason en 1929. Si se toman dos cúmulos de galaxias cualesquiera, la distancia entre ellos crece sin cesar, y lo hace no porque las galaxias o los cúmulos se desplacen, sino porque crece el espacio que media entre ellos. Es más, cuanto más distantes entre sí se encuentren los cúmulos considerados, más veloz es el incremento de la distancia. 
\\

Leavitt estudió las Cefeidas (estrellas de luminosidad variable) aportando datos cruciales para el estudio de Hubble-Lemaitre. Aportando la primera prueba de la expansión del universo. 

La cosmología cientifica nació con la ley de Hubble, la primera observación con significado puramente cosmológico. Hubble obtuvo una relación lineal entre el desplazamiento al rojo $z$ y la distancia $r$:
\begin{equation}
    cz=H_0r
\end{equation}
donde $H_0$ es la constante de Hubble en el momento actual $t_0$, expresada normalmente en $\text{Km s}^{-1}\text{Mpc}^{-1.}$
\\
Podemos calcular el ritmo de expansión actual del universo como 
\begin{equation}
    H_0=70 \frac{\text{km}}{\text{s Mpc}} \times\frac{10^3}{1 \text{km}} \times\frac{1 \text{Mpc}}{10^6 \text{pc}} \times \frac{1 \text{pc}}{3086 \cdot10^{16} \text{m}}=2,3 \cdot 10^{-18} \text{s}^{-1}
\end{equation}
La expansión del universo solo tiene efecto en el vacío, en el momento que este presente alguna fuerza no causa ningun efecto. Aprovechando este contexto vamos a comparar la visión newtoniana con la relativista. 

\\

Usando la formulación \textbf{newtoniana} somos nosotros los que estamos en el centro del universo y las galaxias se desplazan por el espacio alejandose unas de otras lo que provoca el corrimiento al rojo. 
\\

La \textbf{cosmología relativista} establece que todos los puntos del espacio son equivalentes y es el universo el que se expande arrastrando consigo las galaxias. Por lo tanto, el corrimiento al rojo se debe a que la propia expansión del universo hace que aumente la longitud de onda de la luz. Todas las galaxias (salvo el caso particular de Andromeda) sufren este corrimiento al rojo 

\begin{figure}
    \centering
    \includegraphics[width=1\linewidth]{alejamiento.png}
    \caption{Relación entre distancia y velocidad de alejamiento de las estrellas.}
    \label{fig:alejamiento}
\end{figure}


Entonces la expasión ocurre para todo instante $t$ pero no no lo hace al mismo ritmo. 
Para cualquier tiempo $t$ podemos escribir la ley de Hubble. 
\begin{equation}
\dot{R}(t)=H(t)R(t)
\label{ley-Hubble}
\end{equation}
donde la función $R(t)$ nos expresa como se expande el universo y su derivada el ritmo al que lo hace.  Notese que para $t=0$ recuperamos la ley de Hubble-Lemaitre que teniamos originalmente.
\\

$H(t)$ es el parámetro de Hubble que se relaciona con el ritmo de expansión del universo para cada $t$. 
Buscamos obtener una ecuación diferencial para resolver $H(t)$, que surge de la Relatividad General al imponer el principio cosmologíco. 

\begin{definicion}
   El universo es homogéneo (no existe ninguna región preferida) e isótropo (no existe ninguna dirección preferida). Es decir, el universo cumple el \textbf{principio cosmológico}.
\end{definicion}
Consecuencia de este principio cosmológico es que las funciones no dependen de la posición, solo hay dependencia temporal. En cuanto a notación , tomaremos el siguiente criterio, $t=0$ corresponde con el Big-Bang, el origen de tiempo, $t=t_0$ corresponde con el instante actual, es decir la edad del universo. 
\begin{nota}
    Es decir, la expansión ocurre para todo tiempo $t$, pero en cada $t$ ocurre a un ritmo diferente. 
\end{nota}

Si volvemos a la ecuación \ref{ley-Hubble} podemos hacernos la siguiente pregunta. ¿Existe alguna ecuación (diferencial) que permita obtener la función $R(t)$ y por lo tanto calcular $H(t)=\frac{\dot{R}(t)}{R(t)}$?
La respuesta es sí, \textbf{la ecuación de Friedman} que en realidad se obtiene a partir de las ecuaciones de la Relatividad General imponiendo las condiciones de homogeneidad e isotropía, nosotros vamos a obtenerlas aplicando las leyes de Newton. 

\section{Masa y densidad del universo}
La masa total del universo no cambia con el tiempo, pero si lo hace su densidad. Aplicamos la ecuación de continuidad de la masa, suponiendo un universo esférico, podemos escribir el volumen del universo como
\begin{equation*}
    R(t)^3\rho(t)=R_0^3\rho_0
\end{equation*}
De donde obtenemos la siguiente ecuación que relaciona la densidad con el radio del universo
\begin{equation}
    \rho(t)=\frac{R_0^3\rho_0}{R(t)^3}
\end{equation}

$R(t)$ representa la distancia a la ultima galaxia, este razonamiento se puede particularizar para $r(t)$ la distancia a cualquier galaxia. 
Si volvemos a la ecuación 

\subsection{Densidad crítica}
Vamos a escribir la energía mecánica total de una galaxia situada a una distancia $r$ del centro del universo newtoniano, siendo $M$ ka masa contenida en la esfera de radio $r$
\\

Tendremos una energía cinética dada por
\begin{equation*}
    T=\frac{1}{2}mv^2=\frac{1}{2}mH^2r^2
\end{equation*}
y una energía potencial de la forma
\begin{equation}
    U=-G\frac{Mm}{r}=-\frac{4\pi G}{3}r^2 \rho m 
\end{equation}
y entonces la energía mecánica se escribira como
\begin{equation}
    E=T+U=\Big(\frac{1}{2}H^2-\frac{4\pi G}{3}\rho \Big)mr^2
\end{equation}
Para $E=0$ entonces $\rho=\rho_c$ tendremos
\begin{formula}
    \begin{equation}
        \rho_c=\frac{3H^2}{8  \pi G}
    \end{equation}
\end{formula}
para $t=t_0$ tendremos $\rho_c=\frac{3H_0^2}{8  \pi G}$


\section{Ecuación de Friedmann}

Estamos aplicando las leyes de Newton para explicar la expansión del universo, por lo tanto podemos aplicar la segunda ley de Newton, 
\begin{equation}
    F_g=ma=m \ddot{R}(t)
\end{equation}
donde $m$ es la masa de la galaxia más lejana. 
\\
Pero esta masa vemos que no nos afecta ya que,       denotando como $M$ la masa de todo el universo podemso escribir
\begin{equation}
    -G\frac{Mm}{R(t)^2}=m\ddot{R}(t) \rightarrow \ddot{R}(t)=-\frac{GM}{R(t)^2}
\end{equation}
hemos llegado a una \textbf{ecuación diferencial de segundo orden.} Nos surge la siguiente pregunta, ¿podemos convertir esta ecuación diferencial en una de primer orden? La respuesta es sí, usando el factor integrante $\dot{R}$. Entonces del lado izquierdo tenemos que, 
\begin{equation}
    -\frac{GM}{R^2}\dot{R}= GM \frac{d}{dt}\Bigl( \frac{1}{R}\Bigl) =\frac{d}{dt} \Bigl( \frac{GM}{R}\Bigl)
    \label{lado_izquierdo}
\end{equation}
y por el lado de la derecha tenemos: 
\begin{equation}
    \frac{d^2R}{dt^2}\dot{R}= \ddot{R}\dot{R}= \frac{1}{2}\frac{d}{dt}(\dot{R}\dot{R}) = \frac{d}{dt} \Bigl( \frac{\dot{R}^2}2{}\Bigl)
    \label{lado_derecho}
\end{equation}
Si ahora escribimos \ref{lado_izquierdo} e igualamos a \ref{lado_derecho} podemos escribir

\begin{equation*}
\frac{d}{dt} \Bigl( \frac{GM}{R}\Bigl)=\frac{d}{dt} \Bigl( \frac{\dot{R}^2}2{}\Bigl)
\end{equation*}
pasandolo todo al termino de la derecha tenemos que 
\begin{equation}
    0=\frac{d}{dt}\Bigl( \frac{\dot{R^2}}{2}-
    \frac{GM}{R}\Bigl)
    \label{igual_cero}
\end{equation}
De la ecuación \ref{igual_cero} podemos deducir que la resta es constante, y podemos darle el siguiente significado físico 
\begin{formula}
    \begin{equation}
        \dot{R}^2-\frac{2GM}{R} =\text{cte} = kc^2
    \end{equation}
    \label{curvatura}
\end{formula}
el término $k$ lo podemos asociar con la curvatura espacial de universo, y podemos distinguir tres casos, 
\begin{equation}
    \begin{cases}
    k>0 \rightarrow \text{Universo cerrado} 
    \\
    k<0 \rightarrow \text{Universo abierto}
    \\
    k=0 \rightarrow \text{Universo plano }
\end{cases}
\end{equation}
Esta ecuación se suele escribir de dos formas: 
\subsection{Ecuación de Friedmann. Universo solo con materia} 
En este caso sustituimos con la relación obtenida en el tema anterior para la masa,
\begin{equation}
    M=\frac{4}{3} \pi R_0^3 \rho_0
\end{equation} en la ecuación \ref{curvatura} podemos escribir: 
\begin{equation}
    \dot{R}^2 =\frac{8 \pi G R_0^3 \rho_0}{3} \frac{1}{R} -kc^2
\end{equation}
Renombremos al término constante como $D_0$, y llegamos a la ecuación de Friedmann para el universo solo con matería, 
\begin{formula}
    \begin{equation}
    \dot{R}^2 =\frac{D_0}{R}-kc^2   
    \end{equation}
\end{formula}

\subsection{Caso general de las ecuaciones de Friedmann}
Usamos de nuevo la expresión 
\begin{equation}
    M=\frac{4}{3} \pi R(t)^3 \rho(t)
\end{equation}
y sustituimos de forma que llegamos a 
\begin{equation}
    \dot{R}^2 =\frac{2GM}{R} -kc^2 \rightarrow \dot{R}^2 = \frac{8\pi G}{3}\frac{R^3}{R} \rho- kc^2  
\end{equation}
\begin{formula}
    \begin{equation} \Bigl( \frac{\dot{R}}{R}\Bigl)^2=\frac{8 \pi G}{3} \rho(t)-k \Bigl(\frac{c}{R} \Bigl)^2
    \end{equation}
\end{formula}
Para resolver un caso que no sea solo materia necesitamos que nos den información de $\rho(t)$. Veamos un ejemplo, 
\begin{ejemplo}[Universo de solo radiación]
Considere un universo sin curvatura, como indica el fondo cósmico de microondas: 
\begin{itemize}
    \item Partiendo de la ecuación de Friedmann (en su caso general), resuelva la ecuación diferencial que cumplirá $R(t)$ para un universo de radiación, en el que se cumpla la relación 
    \begin{equation}
        \rho(t)R(t)=\rho_0^4R_0^4
        \label{Relación_ejemplo}
    \end{equation}
    \item A partir de la solución obtenida, determine la expresión que tiene la velocidad de expansión $\dot{R}(t)$ y del parámetro de Hubble $H(t)$. Razone si el universo se acelera o se frena. 
\end{itemize}

Para el primer apartado, partimos de la ecuación de Friedmann 
\begin{formula}
       \begin{equation*} \Bigl( \frac{\dot{R}}{R}\Bigl)^2=\frac{8 \pi G}{3} \rho(t)-k \Bigl(\frac{c}{R} \Bigl)^2
    \end{equation*} 
\end{formula}

    el segundo termino se anula ya que $k=0$ (universo sin curvatura) tenemos entonces que: 
    \begin{equation*}
        H^2(t)=\Bigl( \frac{\dot{R(t)}}{R(t)}\Bigl)^2 = \frac{8\pi G}{3} \rho(t)
    \end{equation*}
    y de la relación del enunciado (\ref{Relación_ejemplo}) podemos despejar la densidad $\rho(t)$ de forma que obtenemos
    \begin{equation*}
        \rho(t)=\frac{\rho_0^4 R_0^4}{R(t)^4}
    \end{equation*}
    entonces sustituimos y tenemos que, 
    \begin{equation*}
         H^2(t)=\Bigl( \frac{\dot{R(t)}}{R(t)}\Bigl)^2 = \frac{8\pi G}{3}\frac{\rho_0^4 R_0^4}{R(t)^4}
    \end{equation*}
    podemos pasar $R(t)^2$ al otro término la ecuación se simplifica a 
    \begin{equation*}
    \dot{R}(t)= \frac{8\pi G}{3}\frac{\rho_0^4 R_0^4}{R(t)^2}
    \end{equation*}
    si definimos $D_0$ para juntar todas las constantes, podemos escribir la siguiente ecuación diferencial 
    \begin{equation*}
        \dot{R}(t)^2D_0\frac{1}{R(t)^2}
    \end{equation*}
    podemos ahora simplificarla como
    \begin{equation*}
        \dot{R}(t)^2D_0\frac{1}{R(t)^2} \rightarrow \dot{R(t)}=\sqrt{D_0}\frac{1}{R(t)}
        \end{equation*}
        y si reescribimos como 
        \begin{equation*}
            \frac{dR}{dt}=\sqrt{D_0}\frac{1}{R(t)}
        \end{equation*}
     reorganizando tenemos que
    \begin{equation*}
        RdR=\sqrt{D_0}dt
    \end{equation*}
    que podemos integrar como 
    \begin{equation*}
        \int R dR=\sqrt{D_0}\int dt \rightarrow\frac{R^2}{2}=\sqrt{D_0}\frac{1}t+C 
    \end{equation*}
    despejando $R$ y tomando que para $R(t=0)=0$ y entonces $C=0$ 
    \begin{formula}
    \begin{equation*}
        R^2(t)=2t\sqrt{D_0} \rightarrow R(t)=(D_0)^{1/4} (2t)^{1/2}
    \end{equation*}
    \end{formula}

    llegando a  una expresión para $R(t)$
    \\

    Para el segundo apartado, calculamos $\dot{R}$ derivando en esta ultima expresión y calculamos el parámetro de Hubble, llegaremos a la expresión
    \begin{formula}
            \begin{equation*}
        H(t)=\frac{1}{2}\frac{1}{t}
    \end{equation*}
    \end{formula}
para determinar si el universo se acelera o se frena, bastara con estudiar el signo de la segunda derivada de $R(t)$ respecto del tiempo. 
\end{ejemplo}

\section{Universo de Einstein-de Sitter}

El Universo de Einstein-de Sitter es un modelo del universo propuesto por Albert Einstein y Willem de Sitter en 1932. Al enterarse por primera vez del descubrimiento de Edwin Hubble de una relación lineal entre al corrimiento al rojo de las galaxias y su distancia, Einstein estableció la constante cosmológica en cero en las ecuaciones de Friedmann, lo que resultó en un modelo del universo en expansión conocido como Universo de Friedmann–Einstein. En 1932, Einstein y De Sitter propusieron un modelo cosmico aún más simple al asumir una curvatura espacial que se desvanece, así como una constante cosmológica que se desvanece. En el lenguaje moderno, el universo de Einstein-de Sitter puede describirse como un modelo cosmológico para un universo métrico de Friedmann-Lemaître-Robertson-Walker (FLRW) de solo materia plana, sin curvatura, $k=0$
\\

Usamos la ecuación de Friedmann para el universo de solo materia, anulándose el segundo término,  
\begin{equation*}
    \dot{R}^2= \frac{8 \pi GR_0^3 \rho_0}{3}\frac{1}{R}
\end{equation*}
separamos variables de forma que 
\begin{equation*}
    \dot{R}^2= D_0\frac{1}{R} \rightarrow \Bigl( \frac{dR}{dt}\Bigl)^2=D_0\frac{1}{R} \rightarrow \frac{dR}{dt}=\sqrt{D_0\frac{1}{R}} = D_0^{1/2}\frac{1}{R^{1/2}}
\end{equation*}
y además tenemos que 
\begin{equation*}
    R^{1/2}dR=D_0^{1/2} dt \rightarrow \int R^{1/2}dR = \int D_0^{1/2} dt \rightarrow \frac{2R^{3/2}}{3}=D_0^{1/2}t
\end{equation*}
despejando de este ultimo término llegamos a 
\begin{equation*}
    R^{3/2}=D_0^{1/2}\frac{3t}{2}+C
\end{equation*}
Aplicamos la condición inicial que $R$ en el Big Bang es $0$. Es decir, $R(t=0)=0$. Si particularizamos en $t=0$, podemos dejar $C=0$. 
\\

Por tanto, el universo de Einstein-de Sitter se expande así, 
\begin{equation*}
    R^{3/2}=D_0^{1/2}\frac{3t}{2} \rightarrow R(t)=D_0^{1/3}\Bigl(\frac{3}{2}t \Bigl)
\end{equation*}
y llegamos a la siguiente relación
\begin{formula}
    \begin{equation}
        R(t) \propto t^{2/3}
    \end{equation}
\end{formula}

\section{El modelo Lambda-CDM}
El \textbf{modelo Lambda-CDM}, $\Lambda$-CDM, representa el modelo de concordancia del Big Bang que explica las observaciones cósmicas de la radiación de fondo de microondas, así como la estructura a gran escala del universo y las observaciones realizadas de supernovas, arrojando luz sobre la explicación de la aceleración de la expansión del Universo. Es el modelo conocido más simple que está de acuerdo con todas las observaciones.
\begin{itemize}
    \item $\Lambda$ indica, o hace referencia, a la constante cosmológica como parte de un termino de energía oscura que permite conocer el valor actual de la expansión acelerada del Universo que se inició hace unos 6 mil millones de años. Esta constante fue propuesta por Albert Einstein en 1917 como una modificación de su ecuación del campo gravitatorio, y lo hizo con el objetivo de describir un universo estático, justamente esta modificación resulto en una ecuación que muestra el efecto  contrario, describe un universo que se expande de forma acelerada.
    \\
    La constante cosmológica se describe en términos de $\Omega_{\Lambda}$, la fracción de densidad de energia del universo plano. En la actualidad, $\Omega_{\Lambda} \approx 0.70$ lo que implica que equivale al $70\%$ de la densidad de la energía del universo actual.
    
    \item La materia oscura fría (\textit{Cold Dark Matter}) es el modelo de materia oscura en el que la velocidad de sus partículas es muy inferior a la velocidad de la luz, de ahí el adjetivo fría. La materia oscura fría es no-bariónica, a diferencia de la materia bariónica normal con la que no interacciona excepto mediante la gravedad. Este componente constituye el $26\%$ de la densidad de la energía del actual universo.

    \item El modelo supone un espectro de casi invariancia de escala de perturbaciones primordiales y un universo sin curvatura espacial. También asume que no tiene ninguna topología observable, de modo que el universo es mucho más grande que el horizonte observable de la partícula. Se dan predicciones de inflación cósmica.
\end{itemize}

\begin{figure}[h]
    \centering
    \includegraphics[width=0.75\linewidth]{fondo_cosmico_microondas.png}
    \caption{Las observaciones actuales muestran una anisotropía (muy pequeña) en el fondo cósmico de microondas}
    \label{fig:placeholder}
\end{figure}


\section{Fondo cósmico de Microondas}
La radiación del fondo cósmico de microondas, predicha en 1948, es isotropa hasta en una parte entre $10^5$. El espectrofotómetro FIRAS (en inglés The Far-Infrared Absolute Spectrophotometer), el satélite COBE de la NASA ha medido cuidadosamente el espectro de la radiación de fondo del microondas. El FIRAS comparó el CMB con un cuerpo negro de referencia y no se pudo ver ninguna diferencia en sus espectros.
\\

Esta radiación es una predicción del modelo del Big Bang, ya que según este modelo, el universo primigenio era un plasma compuesto principalmente por electrones, fotones y bariones (protones y neutrones). Los fotones estaban constantemente interactuando con el plasma mediante la dispersión de Thomson, no fue hasta la era del enfriamiento que los bariones comenzaron a interactuar formando otro tipo de materia. 
\begin{thebibliography}{99}

\bibitem{Battaner2010}
E. Battaner, 
\textit{Introducción a la Astrofísica}, 
Alianza Editorial, 2010.

\bibitem{BattanerFlorido2001}
E. Battaner López y E. Florido Navío, 
\textit{100 Problemas de Astrofísica}, 
Alianza Editorial, 2001.

\bibitem{CarrollOstlie2019}
B.~W. Carroll y D.~A. Ostlie, 
\textit{An Introduction to Modern Astrophysics}, 
Cambridge University Press, 2019.

\bibitem{Cepa2023}
J. Cepa, 
\textit{Cosmología física}, 
Ediciones Akal, 2023.

\bibitem{Choudhuri2010}
A.~R. Choudhuri, 
\textit{Astrophysics for physicists}, 
Cambridge University Press, 2010.

\bibitem{Diaz2019}
A.~I. Díaz Beltrán, 
\textit{Estrellas y galaxias}, 
Ediciones Akal, 2019.

\bibitem{Galadi2022}
D. Galadí-Enríquez y J. Gutiérrez Cabello, 
\textit{De la Tierra al universo}, 
Ediciones Akal, 2022.

\bibitem{Geller2019}
R.~M. Geller, R.~A. Freedman y W.~J. Kaufmann III, 
\textit{Universe}, 
W. H. Freeman, 2019.

\bibitem{Karttunen2017}
H. Karttunen, P. Kröger, H. Oja, M. Poutanen y K.~J. Donner (Eds.), 
\textit{Fundamental Astronomy}, 
Springer, 2017.

\bibitem{Maoz2016}
D. Maoz, 
\textit{Astrophysics in a Nutshell}, 
Princeton University Press, 2016.

\bibitem{Martinez2005}
V.~J. Martínez, J.~A. Miralles, E. Marco y D. Galadí-Enríquez, 
\textit{Astronomía fundamental}, 
Universitat de València, 2005.

\bibitem{Unsold2002}
A. Unsöld y B. Baschek, 
\textit{The new cosmos: an introduction to astronomy and astrophysics}, 
Springer Science \& Business Media, 2002.

\end{thebibliography}
\newpage
\end{document}
